\documentclass{article}
\usepackage[utf8]{inputenc}
\usepackage{amssymb,amsmath,amsthm,latexsym}
\begin{document}
\begin{center}
  \section*{PROOF}
\end{center}
  \subsection*{Question}
  Write down the numbers \(1,2,\dots,2n\) where \(n\) is an odd integer. 
  Pick any two numbers \(j,k\) and replace them with \(|j-k|\). Continue 
  this process until only one number remains. Prove that this integer 
  must be odd.
  \subsection*{Answer}
  Consider the possible combinations for \(j\) and \(k\):
  \begin{enumerate}
    \item Without loss of generality, \(j\) is odd and \(k\) is even. 
    \(|j-k|\) is, therefore, odd; reducing the number of evens in 
    the list by one.
    \item In the case where \(j\) and \(k\) are both even, \(|j-k|\) is even. 
    This reduces the number of even numbers in the list by one.
    \item In the case where \(j\) and \(k\) are both odd, \(|j-k|\) is even. 
    This reduces the number of odd numbers in the list by two, and 
    increases the number of even numbers by one.
  \end{enumerate}
  Since case 3 is the only one in which the number of odd integers is 
  reduced in the list, and since that number is reduced by 2 in each 
  application, it must be applied \(l\) times for \(n = 2l + 1\). Leaving one 
  odd integer. Any application of cases 1 or 2 will not result in the 
  reduction of odd numbers. Thus, the remaining integer must be odd, 
  completing the proof.
\end{document}