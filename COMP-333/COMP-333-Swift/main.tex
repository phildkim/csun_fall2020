\documentclass[12pt,a4paper]{article}
\usepackage{mainsty}
\begin{document}
  \begin{center}
    \large\textsc{Algebraic Data-types and Pattern Matching in Swift}
  \end{center}
  \begin{enumerate}
    \item  Define an \imp{enum} named \imp{MyBool} which represents truth and falsehood
    \begin{itemize}
      \item[] \lstinputlisting[language=Swift,firstline=8,lastline=11]{swift/main.swift} 
    \end{itemize}
    \item  Define an \imp{enum} named \imp{MyList} which encodes a singly-linked list of integers, using
    the same cons/nil structure that we used in assignment 1
    \begin{itemize}
      \item[] \lstinputlisting[language=Swift,firstline=1,lastline=4]{swift/main.swift} 
    \end{itemize}
    \item Using the prior enum definition, create a list containing 1, 2, and 3, in that order
    \begin{itemize}
      \item[] \lstinputlisting[language=Swift,firstline=6,lastline=6]{swift/main.swift} 
    \end{itemize}
    \item  Write a function named \imp{length} which takes a list as a parameter, and recursively
    computes the length of the given list.
    \begin{itemize}
      \item[] 
    \end{itemize}
  \end{enumerate}
\end{document}