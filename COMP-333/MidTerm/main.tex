\documentclass[12pt,a4paper]{article}
\usepackage{mainsty}
\begin{document}
  \begin{center}
    \large\textsc{Comp 333 Midterm}
  \end{center}
  \begin{enumerate}
    \item Name one reason why someone might want to use virtual dispatch. % 1
    \begin{itemize}
      \item 
    \end{itemize}
    \item Name one reason why someone might not want to use virtual dispatch. % 2
    \begin{itemize}
      \item
    \end{itemize}
    \clearpage
    \item Consider the following Java code: % 3
    \begin{itemize}
      \item[] \lstinputlisting[language=Java,firstline=2,lastline=12]{java/Main.java}
      \item[] \lstinputlisting[language=Java,firstline=2,lastline=4]{java/I1.java}
      \item[] \lstinputlisting[language=Java,firstline=3,lastline=7]{java/C1.java}
      \item[] \lstinputlisting[language=Java,firstline=3,lastline=7]{java/C2.java}
      \item[] What is the output of the main method? 
      \begin{itemize}
        \item[] c1\\c2
      \end{itemize} 
    \end{itemize}
    \clearpage
    \item Consider the following code snippet: % 4
    \begin{itemize}
      \item[] \lstinputlisting[language=Java,firstline=2,lastline=12]{java/pr4/Main.java}
      \item[] Define any interfaces and/or classes necessary to make this snippet print 8, followed by 2. 
    \end{itemize}
    \clearpage
    \item Consider the following Java code, which simulates a lock which can be either locked or
    unlocked. The lock is an immutable data structure, so locking or unlocking returns a new lock
    in an appropriate state: Refactor this code to use virtual dispatch, instead of using if/else. As a hint, you should have a
    base class/interface for Lock, and subclasses for locked and unlocked locks. (Continued on to
    next page) % 5
    \begin{itemize}
      \item[] \lstinputlisting[language=Java,firstline=3,lastline=29]{java/pr5/Lock.java}
      \item[] Refactor this code to use virtual dispatch, instead of using if/else. As a hint, you should have a
      base class/interface for Lock, and subclasses for locked and unlocked locks. (Continued on to
      next page) 
    \end{itemize}
    \clearpage
    \item The code below does not compile. Why? % 6
    \begin{itemize}
      \item[] \lstinputlisting[language=Java,firstline=3,lastline=10]{java/pr6/MyClass.java}
      \item[] \lstinputlisting[language=Java,firstline=3,lastline=10]{java/pr6/MyInterface.java}
    \end{itemize}
    \clearpage
    \item Java supports sub-typing. Write a Java code snippet that compiles and uses sub-typing. % 7
    \begin{itemize}
      \item 
    \end{itemize} 
    \item Name one reason why someone might prefer static typing over dynamic typing. % 8
    \begin{itemize}
      \item 
    \end{itemize}
    \item Name one reason why someone might prefer dynamic typing over static typing. % 9
    \begin{itemize}
      \item 
    \end{itemize}
    \item Name one reason why someone might prefer strong typing over weak typing. % 10
    \begin{itemize}
      \item 
    \end{itemize}
    \item Name one reason why someone might prefer weak typing over strong typing. % 11
    \begin{itemize}
      \item 
    \end{itemize}
  \end{enumerate}
\end{document}