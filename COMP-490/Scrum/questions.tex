\centerline{\LARGE\textsc{Weekly Questions}}
\centerline{August 31 -\ September 4, 2020}
\textbf{}
\begin{itemize}
  \item[] \textbf{\large Lesson 1:}
  \begin{enumerate}
    \item \textsl{Area this addresses}
    \begin{itemize}
      \item Automated class schedule for CSUN students
    \end{itemize}
    \item \textsl{Market Analysis of topic}
    \begin{itemize}
      \item Automated class schedule in app stores not found
    \end{itemize}
    \item \textsl{Market Research of topic}
    \begin{itemize}
      \item Automated class schedule application not found, but other
      third party software tools are available; such as, selenium,
      robotic process automation, and canvas.
    \end{itemize}
  \end{enumerate} 
\end{itemize}
\textbf{}
\begin{itemize}
  \item[] \textbf{\large Lesson 2:}
  \begin{enumerate}
    \item \textsl{Name the possible patents, licensing, open source status}
    \begin{itemize}
      \item Canvas is a modern, open-source LMS developed and maintained by 
      \href{https://www.instructure.com/canvas/?newhome=canvas}{Instructure Inc.} 
      It is released under the \href{http://www.gnu.org/licenses/agpl-3.0.html}{AGPLv3} 
      license for use by anyone interested in learning more about or using learning 
      management systems.
      \item Cross-platform, Open-Source Robotic Process Automation (RPA)
      \item Selenium project is licensed under the Apache 2.0 license (Copyright)
    \end{itemize}
    \item \textsl{Use some of their code for your solution to make something new/better? Why?}
    \begin{itemize}
      \item Yes, it will save time and money to use third party webtools.
    \end{itemize}
    \item \textsl{Are you able to legally use it beyond research?}
    \begin{itemize}
      \item Yes, but in order to dual-license Canvas product, sign a contributor 
      license agreement (CLA) before accepting a pull request. After submitting 
      a pull request, a status check that indicates if a signature is required 
      or not.
    \end{itemize}
  \end{enumerate} 
\end{itemize}
\textbf{}
\begin{itemize}
  \item[] \textbf{\large Lesson 3:}
  \begin{enumerate}
    \item[1.] \fcolorbox{red}{pink}{\textsl{User Design Workflow}}
  \end{enumerate}   
\end{itemize}
\begin{tikzpicture}
  \begin{umlseqdiag} 
    \umlactor[class=user]{student} 
    % \umlcreatecall[class=B, draw obj=green!70!black, fill obj=green!20, draw=blue!70]{student}{b}
    \umlobject[class=canvas]{login} 
    \umlobject[class=data]{input} 
    \umlobject[class=data]{output} 
    \begin{umlcall}{student}{login}
    \begin{umlcall}{login}{input} 
    \begin{umlcall}{input}{output}
    \begin{umlcall}{output}{student}
    \end{umlcall} 
    \end{umlcall} 
    \end{umlcall} 
    \end{umlcall}
    \umlnote[x=2, y=-5]{input}{user provides data to be automated}
    \umlnote[x=7, y=-5]{output}{notify user upon completion}
  \end{umlseqdiag}
\end{tikzpicture}
\clearpage
\begin{itemize}
  \item[] \textbf{}
  \begin{enumerate}
    \item[2.] \fcolorbox{red}{pink}{\textsl{Describe key mechanic in the Code Design Workflow}} 
    \begin{itemize}
      \item[\$~curl] \texttt{`\url{https://<canvas>/api/v1/accounts/:account_id/logins/:login_id}'}
      \item[\$~curl] \texttt{`\url{https://<canvas>/api/v1/courses/:course_id/enrollments}'}
    \end{itemize}  
  \end{enumerate}  
\end{itemize}
\begin{tikzpicture}[level 1/.style={sibling distance=40mm},edge from parent/.style={->,draw},>=latex]
\node[root] {Canvas API}
  child {node[level 2] (c1) {Login}}
  child {node[level 2] (c2) {Add Course}}
  child {node[level 2] (c3) {Notification}};
\begin{scope}[every node/.style={level 3}]
  \node [below of = c1, xshift=12pt] (c11) {curl `api login'};
  \node [below of = c11] (c12) {curl `api exit'};
  \node [below of = c12] (c13) {curl `api error'};
  \node [below of = c2, xshift=12pt] (c21) {curl `api terms'};
  \node [below of = c21] (c22) {curl `api add'};
  \node [below of = c22] (c23) {curl `api cancel'};
  \node [below of = c23] (c24) {curl `api error'};
  \node [below of = c3, xshift=12pt] (c31) {curl `api notify'};
  \node [below of = c31] (c32) {curl `api error'};
\end{scope}
\foreach \value in {1,2,3}
  \draw[->] (c1.195) |- (c1\value.west);
\foreach \value in {1,...,4}
  \draw[->] (c2.195) |- (c2\value.west);
\foreach \value in {1,2}
  \draw[->] (c3.195) |- (c3\value.west);
\end{tikzpicture}