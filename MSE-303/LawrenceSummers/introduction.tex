\par
After reading the article written by PBS News, \say{Harvard President 
Summers’ Remarks About Women in Science, Engineering}~\cite{pbsnewshour}, 
it's hard not to have some sort of emotions whether positive or negative 
this will be discussed later. First, breaking down and analyzing his 
comments is essential in order to come to a conclusion on the former 
President Lawerence Summers. As described by Summers himself, 
attempt to adopt an entirely positive, rather than normative
approach, and just try to think about and offer some hypotheses,
why women are underrespresented in the science and engineering field.
He gave three generally broad hypotheses in respectively orders:
(1) high-powered job hypotheses, (2) different availability of aptitude 
at the high end, and (3) different socialization and patterns of 
discrimination in a search.

\par
As Summers described, it is a \say{positive understanding} that high-powered jobs
are not only disired by men but because of society, womens priority is 
getting pregnant and starting a family.

\par
\say{Innately thus genetically different from men and lacked what he characterized as 
natural ability of men in these fields,~\cite{youtube}} Susan reads out loud 
quoting what Summers had said in 2005. This very controversy hypotheses claimed that
men out-performed women on the lower and upper ends of the test score spectrum.

\par
Lastly, different socialization and patterns of discriimination, which generally
gravitates towards nature vs.\ nurture.