
\par
The Nobel Prize started in the early 19th century because of a gentleman by the name of Alfred Nobel who had 
substantial worth due to his many inventions. His interests in physics, chemistry, physiology, literature, 
and peace work are the reasons why the Nobel Prize is divided into five different categories. As one of his 
wills before he passed away, his wish was to give out his fortune as prizes to those who can bestow the 
greatest among humankind in the five categories he was most interested in. This prestigious ceremony overtime 
has been awarded to 950 people and organizations, however, there is a significant difference in the number of 
women laureates compared to men laureates.

\par
I had the impression that without a doubt the Nobel Prize always goes to the scientist with the most groundbreaking 
discovery for the advancement of humanity. It turns out there’s more than just pure science, politics and fame has 
fallen into the mix of this prestigious prize, which seems to be the main reasoning why women are underrepresented 
in the Nobel Prize awards. In the case of both Marie Curie and Lisa Meitner, they are both known to be reserved in 
their temperament. Even with fame, Marie Curie did not view her discoveries as a means to make money, but rather 
they \say{must belong to science.}\cite{kopievanpbs}
