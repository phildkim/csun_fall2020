\documentclass[12pt,a4paper]{article}
\usepackage{mainsty}
\begin{document}
  \begin{center}
    \large\textsc{MSE Final}
  \end{center}

\begin{enumerate}
  \item \textbf{Women's place during and after WWII.}
  \begin{itemize}
    \item[] 
  \end{itemize}
  \item \textbf{Women's contributions during the Manhattan Project, as well as timeline of prominent women leading up to the discovery of the bomb.}
  \begin{itemize}
    \item[] \imp{July 1898} -\ French scientists Marie Curie and Pierre Curie publish their discovery of radium in Comptes Rendus de l'Académie des Sciences. In the December issue of the same journal, the Curies announce their discovery of polonium, named after Marie's homeland of Poland. The Curies in 1903 share the Nobel Prize in physics with Antoine Henri Becquerel; Marie is the first woman scientist to win the Nobel Prize.
    \item[] \imp{1906 -\ 1932} -\ Physicists and other scientists delve into the complexities of the atom, determined to understand its structure and to gain the ability to break it apart.
    \item[] \imp{January 1934} -\ French scientists Frédéric and Irène Joliot-Curie, daughter of Marie Curie, publish their discovery of artificial radioactivity in Comptes Rendus de l'Académie des Sciences. The husband and wife team bombard boron, aluminum, and magnesium to produce isotopes of these elements not found naturally. This discovery leads to the production of cheap and plentiful radioactive materials for medical purposes.
    \item[] \imp{Feburary 1939} -\ Scientists Lise Meitner and Otto Frisch, her nephew, publish a theoretical interpretation of the Hahn-Strassmann results in Nature. In the letter, titled ``Disintegration of Uranium by Neutrons: a New Type of Nuclear Reaction\,'' Meitner and Frisch introduce the term ``fission'' to describe the splitting of a nucleus to produce energy in a nuclear chain reaction.
    \item[] \imp{1941 -\ 1945} -\ Maria Goeppert-Mayer works on the Manhattan Project at Columbia University, working in the Substitute Alloy Materials Laboratory performing uranium isotope separation experiments. Maria is a German-born American theoretical physicist and goes on to win the Nobel Prize in 1963 for her nuclear shell model.
    \item[] \imp{Decemeber 6, 1941} -\ President Franklin D. Roosevelt authorizes the Manhattan Engineering District (later called the Manhattan Project) with a \$2 billion appropriation to build an atomic bomb. Thousands of women were essential throughout the entire project to its success.
    \item[] \imp{1942} -\ Along with their kids, Nella and Giulio, Laura Fermi and her husband Enrico Fermi move to Chicago for her husband's work at the Metallurgical Laboratory at the University of Chicago. Laura and her husband have a ``voluntary system of censorship'' about his work; Laura doesn't ask questions and Enrico doesn't offer information. Laura becomes a regular host for the Met Lab workers, and volunteers with the Red Cross. Her family later moves to Spartan Army housing in Los Alamos, New Mexico, in 1944, where she works in the Health Group taking blood counts. She published Atoms in the Family in 1954 and five other books during her life.
    \item[] \imp{1942} -\ At age 23, Leona Woods Marshall Libby, is the youngest and only female team member to build and conduct an experiment with the world's first nuclear reactor pile, alongside Enrico Fermi. Leona is the only woman present when the reactor goes critical, and is instrumental in the construction and utilization of Geiger counters for the experiment analysis. Later on she helps solve the problem of xenon poisoning at the Hanford plutonium production site, and oversees production of the plutonium reactors at Hanford.
    \item[] \imp{1942} -\ A University of Pennsylvania alumna, Charlotte Serber, becomes the first female division leader at Los Alamos Laboratory. Charlotte was the head of the Library Division, and responsible for procuring books and resources, organizing, and managing the library of reference books and scholarly journals on physics, chemistry, engineering, and metallurgy. Charlotte built the Library from zero books to over three thousand, and oversaw staff to type, edit, and reproduce the work of the Lab technicians. She was the keeper of secrets -\ and had no previous library experience when she began this undertaking.
    \item[] \imp{1943} -\ Grade school teacher Elise Novy founds a Girl Scouts troop in the Secret City of Oak Ridge. This is the first youth organization allowed on site. In 2018, Girl Scouts and Oak Ridge is celebrating the 75 anniversary of this partnership
    \item[] \imp{1943 -\ 1946} -\ Moving with her husband Robert to Los Alamos, Ruth Marshak, knows nothing about the town, what work her husband is doing, or how long they will be staying. Ruth works in the Housing Office at Los Alamos and teaches third grade at the Site Y's Central School, where experimental blasts in nearby canyons that shake the foundation of the school and frighten the students. Ruth later organizes a group of women to write short stories about the project, which become the book Standing by and Making Do:\ Women of Wartime Los Alamos (published in 1987).
    \item[] \imp{1943} -\ General Leslie Groves requests a detachment of the Women's Army Corps be assigned to the Manhattan Engineer District, to provide military personnel to handle the heavy load of classified and sensitive mail and records relating to the Project. By the end of the war, more than 400 members of the Women's Army Corps served in the Manhattan Engineer District.
    \item[] \imp{1944} -\ A refugee of Nazi Germany, Lilli Hornig, moves to Los Alamos with her husband, Don. Even though Lilli has a graduate degree in chemistry from Harvard, she initially is offered a job as a typist. Lilli starts working on plutonium chemistry, and later transfers to the explosives group when concerns arise about potential reproductive damage from exposure to plutonium. She signs a petition advocating demonstrating the atomic bomb's destructive power as a warning to Imperial Japan, rather than dropping it on civilian populations there. Lilli went on to found Higher Education Resource Services, a nonprofit working to advance females pursing higher education and wrote/edited three books on women in science.
    \item[] \imp{1944} -\ Toni Oppenheimer, daughter of J. Robert Oppenheimer, Director of the Los Alamos National Laboratory, and Katherine ``Kitty'' Oppenheimer, is born at Los Alamos. Security requires that Toni's birth certificate list the anonymous ``P.O. Box 1663'' as her birthplace. She is one of many children born at the Manhattan Project sites, and is on site until age three when her father becomes the Director of the Institute for Advanced Study in Princeton.
    \item[] \imp{1945} -\ Floy Agnes ``Aggie'' Lee graduates from the University of New Mexico and moves to Los Alamos to join the hematology laboratory. Aggie collects blood from the researchers, and analyzes the blood cell information. She frequently plays Enrico Fermi in tennis after work, without knowing who he is, as all employees are referred to by employee number instead of by name for secrecy. Aggie's parents are Pueblo Indian and White, and she is one of the few Native Americans working on the Manhattan Project. Later, she works at Argonne National Laboratory and earns her Ph.D., all while being a single mom.
    \item[] \imp{July 1945} -\ A Swarthmore graduate, Frances Dunne, is recruited from her job at Kirtland Air Force Base to begin work at Los Alamos with the Explosives Assembly Group. She is the only woman on the team, and part of the assembly crew for the Trinity test, the world's first nuclear explosion, which explodes on July 16, 1945.
    \item[] \imp{1966} -\ Lise Meitner is the first woman to receive the Enrico Fermi Award for her pioneering research in naturally occurring radioactivity and extensive experimental studies leading to the discovery of fission. The Enrico Fermi Award is one of the oldest and most prestigious science and technology honors bestowed by the federal government. 
  \end{itemize}
  \item \textbf{Prominent and deserving women who were left out of the Nobel Prize and why. As well as relationships seen among those who have won the prize.}
  \begin{itemize}
    \item[]
  \end{itemize}
  \item \textbf{The Civil Rights Act of 1964 and gender role.}
  \begin{itemize}
    \item[] 
  \end{itemize}
  \item \textbf{Feminism and the women’s liberation movement -\ how the perceptions of women's role in science changed.}
  \begin{itemize}
    \item[] 
  \end{itemize}
  \item \textbf{Topics covered in Why so Few?}
  \begin{itemize}
    \item[] 
  \end{itemize}
  \item \textbf{How the number of women in STEM has changed through the years. Which fields have seen growth/drop?}
  \begin{itemize}
    \item[] 
  \end{itemize}
  \item \textbf{Title IX.}
  \begin{itemize}
    \item[] 
  \end{itemize}
  \item \textbf{Women's contributions to the ENIAC.}
  \begin{itemize}
    \item[] 
  \end{itemize}
  \item \textbf{Women's contributions to the National Bureau of Standards.}
  \begin{itemize}
    \item[] 
  \end{itemize}
  \item \textbf{Women in computing.}
  \begin{itemize}
    \item[] 
  \end{itemize}
  \item \textbf{Increasing African American Women to Engineering.}
  \begin{itemize}
    \item[] 
  \end{itemize}
  \item \textbf{BBC Documentary: Calculating Ada The Countess of Computing.}
  \begin{itemize}
    \item[] 
  \end{itemize}
  \item \textbf{Why are there so few women in engineering and computing?}
  \begin{itemize}
    \item[] 
  \end{itemize}
\end{enumerate}
  

\clearpage
\begin{center}
  \large\textsc{Notable Women}
\end{center}

% \link{https://www.nytimes.com/2019/08/28/obituaries/elizabeth-rona-overlooked.html} 
\begin{enumerate}
  \item \imp{Elizabeth Rona} -\ 
  \item \imp{Leona Libby} -\
  \item \imp{Cecilia Payne-Gaposchkin} -\
  \item \imp{Maria Goeppert Mayer} -\
  \item \imp{Chien-Shiung Wu} -\
  \item \imp{Maria Mayer} -\
  \item \imp{Gerty Cori} -\
  \item \imp{Dorothy Crowfoot Hodgkin} -\
  \item \imp{Rosalind Franklin} -\
  \item \imp{Rosalyn Sussman Yalow} -\
  \item \imp{Fay Ajzenberg-Selove's} -\
  \item \imp{Barbara McClintock} -\
  \item \imp{Jane Goodall} -\
  \item \imp{Diane Fossey} -\
  \item \imp{Birute Galdikas} -\
  \item \imp{Louis Leakey} -\
  \item \imp{Margarett Hamilton} -\ In 1958, received her undergraduate degree in mathematics and was married shortly after. She put graduate school on hold, and got a job at MIT as a programmer to support her husband while he pursued his law degree at Harvard. She learned to predict weather and detect enemy planes which got her a job at NASA.\ She discovered a flaw on rocket because her daughter was playing in the simulator and pressed a button, so she programmed the software to correct for this error just in case the astronauts would do the same. July 20th 1969, Apollo 11 was able to land on the moon thanks to her software program. Software programming is popularized because of her. \link{https://www.wired.com/2015/10/margaret-hamilton-nasa-apollo/}
  \item \imp{Sally Ride} -\
  \item \imp{Rachel Carson} -\
  \item \imp{Hedy Lamarr} -\
  \item \imp{Grace Hopper} -\
  \item \imp{Ada Lovelace} -\
  \item \imp{Katherine Johnson} -\ Katherine G. Johnson is a pioneer in American space history. A NASA mathematician, Johnson's computations have influenced every major space program from Mercury through the Shuttle. She even calculated the flight path for the first American mission space. Born in 1918 in West Virginia, Johnson was a talented student who entered college at only 15 years old. At West Virginia State University, W.W. Schiefflin Clayor, the third African American to earn a PhD. in mathematics, recognized Johnson's abilities and motivated her to take advanced math. Johnson would go on to earn a graduate degree in mathematics. In 1953, Johnson was contracted as a research mathematician at the Langley Research Center with the National Advisory Committee for Aeronautics, the agency that preceded NASA.\ She worked in a pool of women performing math calculations until she was temporarily assigned to help the all male flight research team and wound up staying there. Johnson's specialty was calculating the trajectories for space shots which determined the timing for launches, including the Mercury mission and Apollo 11, the mission to the moon.
  \item \imp{Paula Hammond} -\
  \item \imp{Lydia Villa-Komaroff} -\
  \item \imp{Catherine Wolf} -\
  \item \imp{Mildred Dresselhaus} -\
  \item \imp{Marie Tharp} -\
  \item \imp{Cady Coleman} -\ As a chemist and an astronaut, Cady Coleman boasts an impressive list of accolades that orbit around science and space. The former United States Air Force officer has logged nearly 4,500 hours and 180 days in space as a NASA astronaut. She's a veteran of two space shuttle missions, and participated in a six-month tour on the International Space Station. Coleman is also an American chemist with a doctorate in polymer science and engineering from University of Massachusetts Amherst and a Bachelor's in chemistry from Massachusetts Institute of Technology. While completing work for her Ph.D. at UMASS, she joined the U.S. Air Force as Second Lieutenant. She retired from the Air Force in 2009. Coleman later served as a liaison to NASA's newest Commercial Space partners where she helps assemble and integrate supply ship operations aboard the International Space Station. She retired from NASA in 2016.
  \item \imp{Ellen Ochoa} -\
  \item \imp{Edith Clark} -\
\end{enumerate}


\end{document}