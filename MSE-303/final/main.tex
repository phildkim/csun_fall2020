\documentclass[12pt,a4paper]{article}
\usepackage{mainsty}
\begin{document}
  \begin{center}
    \large\link{MSE Final}
  \end{center}
  \begin{enumerate}
    \item \textbf{Women's place during and after WWII.}
    \begin{itemize}
      \item \imp{Europe before WWII}
      \begin{itemize}
        \item \sub{Marie Curie}: the first woman to win a Nobel prize in 1903 (physics), went on to become a double Nobel prize winner in 1911, both for her work on radiation. She was the first person to win two Nobel prizes, a feat accomplished by only three others since then. She also was the first woman to teach at Sorbonne University in Paris, France.
        \item \sub{Lise Meitner}: played a major role in the discovery of nuclear fission. As head of the physics section at the Kaiser Wilhelm Institute in Berlin she collaborated closely with the head of chemistry Otto Hahn on atomic physics until forced to flee Berlin in 1938. In 1939, in collaboration with her nephew Otto Frisch, Meitner derived the theoretical explanation for an experiment performed by Hahn and Fritz Strassman in Berlin, thereby demonstrating the occurrence of nuclear fission. The possibility that Fermi's bombardment of uranium with neutrons in 1934 had instead produced fission by breaking up the nucleus into lighter elements, had actually first been raised in print in 1934, by chemist Ida Noddack (co-discover of the element rhenium), but this suggestion had been ignored at the time, as no group made a concerted effort to find any of these light radioactive fission products.
        \item \sub{Emmy Noether}: revolutionized abstract algebra, filled in gaps in relativity, and was responsible for a critical theorem about conserved quantities in physics. One notes that the Erlangen program attempted to identify invariants under a group of transformations. On 16 July 1918, before a scientific organization in Göttingen, Felix Klein read a paper written by Emmy Noether, because she was not allowed to present the paper herself. In particular, in what is referred to in physics as Noether's theorem, this paper identified the conditions under which the Poincaré group of transformations (now called a gauge group) for general relativity defines conservation laws. Noether's papers made the requirements for the conservation laws precise. Among mathematicians, Noether is best known for her fundamental contributions to abstract algebra, where the adjective noetherian is nowadays commonly used on many sorts of objects.
        \item \sub{Mary Cartwright}: was a British mathematician who was the first to analyze a dynamical system with chaos. Inge Lehmann, a Danish seismologist, first suggested in 1936 that inside the Earth's molten core there may be a solid inner core. Women such as Margaret Fountaine continued to contribute detailed observations and illustrations in botany, entomology, and related observational fields. Joan Beauchamp Procter, an outstanding herpetologist, was the first woman Curator of Reptiles for the Zoological Society of London at London Zoo.
        \item \sub{Florence Sabin}: was an American medical scientist. Sabin was the first woman faculty member at Johns Hopkins in 1902, and the first woman full-time professor there in 1917. Her scientific and research experience is notable. Sabin published over 100 scientific papers and multiple books.
      \end{itemize}
      \item \imp{US before WWII}
      \begin{itemize}
        \item Women moved into science in significant numbers by 1900, helped by the women's colleges and by opportunities at some of the new universities. Margaret Rossiter's books Women Scientists in America: Struggles and Strategies to 1940 and Women Scientists in America: Before Affirmative Action 1940–1972 provide an overview of this period, stressing the opportunities women found in separate women's work in science.
        \item In 1892, \sub{Ellen Swallow Richards} called for the ``christening of a new science'' – ``oekology'' (ecology) in a Boston lecture. This new science included the study of ``consumer nutrition'' and environmental education. This interdisciplinary branch of science was later specialized into what is currently known as ecology, while the consumer nutrition focus split off and was eventually relabeled as home economics, which provided another avenue for women to study science. Richards helped to form the American Home Economics Association, which published a journal, the Journal of Home Economics, and hosted conferences. Home economics departments were formed at many colleges, especially at land grant institutions. In her work at MIT, Ellen Richards also introduced the first biology course in its history as well as the focus area of sanitary engineering.
        \item Women also found opportunities in botany and embryology. In psychology, women earned doctorates but were encouraged to specialize in educational and child psychology and to take jobs in clinical settings, such as hospitals and social welfare agencies.
        \item In 1901, \sub{Annie Jump Cannon} first noticed that it was a star's temperature that was the principal distinguishing feature among different spectra. This led to re-ordering of the ABC types by temperature instead of hydrogen absorption-line strength. Due to Cannon's work, most of the then-existing classes of stars were thrown out as redundant. Afterward, astronomy was left with the seven primary classes recognized today, in order: O, B, A, F, G, K, M, that has since been extended.
        \item \sub{Henrietta Swan Leavitt} made fundamental contributions to astronomy. Henrietta Swan Leavitt first published her study of variable stars in 1908. This discovery became known as the ``period-luminosity relationship'' of Cepheid variables. Our picture of the universe was changed forever, largely because of Leavitt's discovery.
        \item \sub{Edwin Hubble}, renowned American astronomer, were made possible by Leavitt's groundbreaking research and Leavitt's Law. ``If Henrietta Leavitt had provided the key to determine the size of the cosmos, then it was Edwin Powell Hubble who inserted it in the lock and provided the observations that allowed it to be turned'', wrote David H. and Matthew D.H. Clark in their book Measuring the Cosmos. Hubble often said that Leavitt deserved the Nobel for her work. Gösta Mittag-Leffler of the Swedish Academy of Sciences had begun paperwork on her nomination in 1924, only to learn that she had died of cancer three years earlier (the Nobel prize cannot be awarded posthumously).
        \item In 1925, Harvard graduate student \sub{Cecilia Payne-Gaposchkin} demonstrated for the first time from existing evidence on the spectra of stars that stars were made up almost exclusively of hydrogen and helium, one of the most fundamental theories in stellar astrophysics.
        \item Canadian born \sub{Maud Menten} worked in the US and Germany. Her most famous work was on enzyme kinetics together with Leonor Michaelis, based on earlier findings of Victor Henri. This resulted in the Michaelis–Menten equations. Menten also invented the azo-dye coupling reaction for alkaline phosphatase, which is still used in histochemistry. She characterised bacterial toxins from B. paratyphosus, Streptococcus scarlatina and Salmonella ssp., and conducted the first electrophoretic separation of proteins in 1944. She worked on the properties of hemoglobin, regulation of blood sugar level, and kidney function.
        \item World War II brought some new opportunities. The Office of Scientific Research and Development, under Vannevar Bush, began in 1941 to keep a registry of men and women trained in the sciences. Because there was a shortage of workers, some women were able to work in jobs they might not otherwise have accessed. Many women worked on the Manhattan Project or on scientific projects for the United States military services. Women who worked on the Manhattan Project included \sub{Leona Woods Marshall}, \sub{Katharine Way}, and \sub{Chien-Shiung Wu}.
        \item Women in other disciplines looked for ways to apply their expertise to the war effort. Three nutritionists, \sub{Lydia J. Roberts, Hazel K. Stiebeling}, and \sub{Helen S. Mitchell}, developed the Recommended Dietary Allowance in 1941 to help military and civilian groups make plans for group feeding situations. The RDAs proved necessary, especially, once foods began to be rationed. \sub{Rachel Carson} worked for the United States Bureau of Fisheries, writing brochures to encourage Americans to consume a wider variety of fish and seafood. She also contributed to research to assist the Navy in developing techniques and equipment for submarine detection.
        \item Women in psychology formed the National Council of Women Psychologists, which organized projects related to the war effort. The NCWP elected \sub{Florence Laura Goodenough} president. In the social sciences, several women contributed to the Japanese Evacuation and Resettlement Study, based at the University of California. This study was led by sociologist \sub{Dorothy Swaine Thomas}, who directed the project and synthesized information from her informants, mostly graduate students in anthropology. These included \sub{Tamie Tsuchiyama}, the only Japanese-American woman to contribute to the study, and \sub{Rosalie Hankey Wax}.
        \item \sub{Mary Sears}: a planktonologist, researched military oceanographic techniques as head of the Hydgrographic Office's Oceanographic Unit.
        \item \sub{Florence Straten}: a chemist, worked as an aerological engineer. She studied the effects of weather on military combat.
        \item \sub{Grace Hopper}: a mathematician, became one of the first computer programmers for the Mark I computer.
        \item \sub{Mina Spiegel Rees}: also a mathematician, was the chief technical aide for the Applied Mathematics Panel of the National Defense Research Committee. 
        \item \sub{Gerty Cori} was a biochemist who discovered the mechanism by which glycogen, a derivative of glucose, is transformed in the muscles to form lactic acid, and is later reformed as a way to store energy. For this discovery she and her colleagues were awarded the Nobel prize in 1947, making her the third woman and the first American woman to win a Nobel Prize in science. She was the first woman ever to be awarded the Nobel Prize in Physiology or Medicine. Cori is among several scientists whose works are commemorated by a U.S. postage stamp.
      \end{itemize}
      \item \imp{US after WWII}
      \begin{itemize}
        \item \sub{Kay McNulty}, \sub{Betty Jennings}, \sub{Betty Snyder},\\ \sub{Marlyn Wescoff}, \sub{Fran Bilas} and \sub{Ruth Lichterman} were six of the original programmers for the ENIAC, the first general purpose electronic computer.
        \item \sub{Linda B. Buck} is a neurobiologist who was awarded the 2004 Nobel Prize in Physiology or Medicine along with Richard Axel for their work on olfactory receptors.
        \item Biologist and activist \sub{Rachel Carson} published Silent Spring, a work on the dangers of pesticides, in 1962.
        \item \sub{Eugenie Clark}, popularly known as The Shark Lady, was an American ichthyologist known for her research on poisonous fish of the tropical seas and on the behavior of sharks.
        \item \sub{Ann Druyan} is an American writer, lecturer and producer specializing in cosmology and popular science. Druyan has credited her knowledge of science to the 20 years she spent studying with her late husband, \sub{Carl Sagan}, rather than formal academic training. She was responsible for the selection of music on the Voyager Golden Record for the Voyager 1 and Voyager 2 exploratory missions. Druyan also sponsored the Cosmos 1 spacecraft.
        \item \sub{Gertrude Elion} was an American biochemist and pharmacologist, awarded the Nobel Prize in Physiology or Medicine in 1988 for her work on the differences in biochemistry between normal human cells and pathogens.
        \item \sub{Sandra Faber}, with \sub{Robert Jackson}, discovered the Faber–Jackson relation between luminosity and stellar dispersion velocity in elliptical galaxies. She also headed the team which discovered the Great Attractor, a large concentration of mass which is pulling a number of nearby galaxies in its direction.
        \item Zoologist \sub{Dian Fossey} worked with gorillas in Africa from 1967 until her murder in 1985.
        \item Astronomer \sub{Andrea Ghez} received a MacArthur ``genius grant'' in 2008 for her work in surmounting the limitations of earthbound telescopes.
        \item \sub{Maria Goeppert-Mayer} was the second female Nobel Prize winner in Physics, for proposing the nuclear shell model of the atomic nucleus. Earlier in her career, she had worked in unofficial or volunteer positions at the university where her husband was a professor. Goeppert-Mayer is one of several scientists whose works are commemorated by a U.S. postage stamp.
        \item \sub{Sulamith Low Goldhaber} and her husband Gerson Goldhaber formed a research team on the K meson and other high-energy particles in the 1950s.
        \item \sub{Carol Greider} and the Australian born \sub{Elizabeth Blackburn}, along with \sub{Jack W. Szostak}, received the 2009 Nobel Prize in Physiology or Medicine for the discovery of how chromosomes are protected by telomeres and the enzyme telomerase.
        \item Rear Admiral \sub{Grace Murray Hopper} developed the first computer compiler while working for the Eckert Mauchly Computer Corporation, released in 1952.
        \item \sub{Deborah Jin's} team at JILA, in Boulder, Colorado in 2003 produced the first fermionic condensate, a new state of matter.
        \item \sub{Stephanie Kwolek}, a researcher at DuPont, invented poly-paraphenylene terephthalamide – better known as Kevlar.
        \item \sub{Lynn Margulis} is a biologist best known for her work on endosymbiotic theory, which is now generally accepted for how certain organelles were formed.
        \item \sub{Barbara McClintock's} studies of maize genetics demonstrated genetic transposition in the 1940s and 1950s. She dedicated her life to her research, and she was awarded the Nobel Prize in Physiology or Medicine in 1983. McClintock is one of several scientists whose works are commemorated by a U.S. postage stamp.
        \item \sub{Nita Ahuja} is a renowned surgeon-scientist known for her work on CIMP in cancer, she is currently the Chief of surgical oncology at Johns Hopkins Hospital. First woman ever to be the Chief of this prestigious department.
        \item \sub{Carolyn Porco} is a planetary scientist best known for her work on the Voyager program and the Cassini–Huygens mission to Saturn. She is also known for her popularization of science, in particular space exploration.
        \item Physicist \sub{Helen Quinn}, with \sub{Roberto Peccei}, postulated Peccei-Quinn symmetry. One consequence is a particle known as the axion, a candidate for the dark matter that pervades the universe. Quinn was the first woman to receive the Dirac Medal and the first to receive the Oskar Klein Medal.
        \item \sub{Lisa Randall} is a theoretical physicist and cosmologist, best known for her work on the Randall–Sundrum model. She was the first tenured female physics professor at Princeton University.
        \item \sub{Sally Ride} was an astrophysicist and the first American woman, and then-youngest American, to travel to outer space. Ride wrote or co-wrote several books on space aimed at children, with the goal of encouraging them to study science. Ride participated in the Gravity Probe B (GP-B) project, which provided more evidence that the predictions of Einstein's general theory of relativity are correct.
        \item Through her observations of galaxy rotation curves, astronomer Vera Rubin discovered the Galaxy rotation problem, now taken to be one of the key pieces of evidence for the existence of dark matter. She was the first female allowed to observe at the Palomar Observatory.
        \item \sub{Sara Seager} is a Canadian-American astronomer who is currently a professor at the Massachusetts Institute of Technology and known for her work on extrasolar planets.
        \item Astronomer \sub{Jill Tarter} is best known for her work on the search for extraterrestrial intelligence. Tarter was named one of the 100 most influential people in the world by Time Magazine in 2004. She is the former director of SETI.\
        \item \sub{Rosalyn Yalow} was the co-winner of the 1977 Nobel Prize in Physiology or Medicine ``together with Roger Guillemin and Andrew Schally'' for development of the radioimmunoassay technique.
      \end{itemize}
    \end{itemize}
    \item \textbf{Women's contributions during the Manhattan Project, as well as timeline of prominent women leading up to the discovery of the bomb.}
    \begin{itemize}
      \item[] \imp{July 1898} -\ French scientists Marie Curie and Pierre Curie publish their discovery of radium in Comptes Rendus de l'Académie des Sciences. In the December issue of the same journal, the Curies announce their discovery of polonium, named after Marie's homeland of Poland. The Curies in 1903 share the Nobel Prize in physics with Antoine Henri Becquerel; Marie is the first woman scientist to win the Nobel Prize.
      \item[] \imp{1906 -\ 1932} -\ Physicists and other scientists delve into the complexities of the atom, determined to understand its structure and to gain the ability to break it apart.
      \item[] \imp{January 1934} -\ French scientists Frédéric and Irène Joliot-Curie, daughter of Marie Curie, publish their discovery of artificial radioactivity in Comptes Rendus de l'Académie des Sciences. The husband and wife team bombard boron, aluminum, and magnesium to produce isotopes of these elements not found naturally. This discovery leads to the production of cheap and plentiful radioactive materials for medical purposes.
      \item[] \imp{February 1939} -\ Scientists Lise Meitner and Otto Frisch, her nephew, publish a theoretical interpretation of the Hahn-Strassmann results in Nature. In the letter, titled ``Disintegration of Uranium by Neutrons: a New Type of Nuclear Reaction\,'' Meitner and Frisch introduce the term ``fission'' to describe the splitting of a nucleus to produce energy in a nuclear chain reaction.
      \item[] \imp{1941 -\ 1945} -\ Maria Goeppert-Mayer works on the Manhattan Project at Columbia University, working in the Substitute Alloy Materials Laboratory performing uranium isotope separation experiments. Maria is a German-born American theoretical physicist and goes on to win the Nobel Prize in 1963 for her nuclear shell model.
      \item[] \imp{December 6, 1941} -\ President Franklin D. Roosevelt authorizes the Manhattan Engineering District (later called the Manhattan Project) with a \$2 billion appropriation to build an atomic bomb. Thousands of women were essential throughout the entire project to its success.
      \item[] \imp{1942} -\ Along with their kids, Nella and Giulio, Laura Fermi and her husband Enrico Fermi move to Chicago for her husband's work at the Metallurgical Laboratory at the University of Chicago. Laura and her husband have a ``voluntary system of censorship'' about his work; Laura doesn't ask questions and Enrico doesn't offer information. Laura becomes a regular host for the Met Lab workers, and volunteers with the Red Cross. Her family later moves to Spartan Army housing in Los Alamos, New Mexico, in 1944, where she works in the Health Group taking blood counts. She published Atoms in the Family in 1954 and five other books during her life.
      \item[] \imp{1942} -\ At age 23, Leona Woods Marshall Libby, is the youngest and only female team member to build and conduct an experiment with the world's first nuclear reactor pile, alongside Enrico Fermi. Leona is the only woman present when the reactor goes critical, and is instrumental in the construction and utilization of Geiger counters for the experiment analysis. Later on she helps solve the problem of xenon poisoning at the Hanford plutonium production site, and oversees production of the plutonium reactors at Hanford.
      \item[] \imp{1942} -\ A University of Pennsylvania alumna, Charlotte Serber, becomes the first female division leader at Los Alamos Laboratory. Charlotte was the head of the Library Division, and responsible for procuring books and resources, organizing, and managing the library of reference books and scholarly journals on physics, chemistry, engineering, and metallurgy. Charlotte built the Library from zero books to over three thousand, and oversaw staff to type, edit, and reproduce the work of the Lab technicians. She was the keeper of secrets -\ and had no previous library experience when she began this undertaking.
      \item[] \imp{1943} -\ Grade school teacher Elise Novy founds a Girl Scouts troop in the Secret City of Oak Ridge. This is the first youth organization allowed on site. In 2018, Girl Scouts and Oak Ridge is celebrating the 75 anniversary of this partnership
      \item[] \imp{1943 -\ 1946} -\ Moving with her husband Robert to Los Alamos, Ruth Marshak, knows nothing about the town, what work her husband is doing, or how long they will be staying. Ruth works in the Housing Office at Los Alamos and teaches third grade at the Site Y's Central School, where experimental blasts in nearby canyons that shake the foundation of the school and frighten the students. Ruth later organizes a group of women to write short stories about the project, which become the book Standing by and Making Do:\ Women of Wartime Los Alamos (published in 1987).
      \item[] \imp{1943} -\ General Leslie Groves requests a detachment of the Women's Army Corps be assigned to the Manhattan Engineer District, to provide military personnel to handle the heavy load of classified and sensitive mail and records relating to the Project. By the end of the war, more than 400 members of the Women's Army Corps served in the Manhattan Engineer District.
      \item[] \imp{1944} -\ A refugee of Nazi Germany, Lilli Hornig, moves to Los Alamos with her husband, Don. Even though Lilli has a graduate degree in chemistry from Harvard, she initially is offered a job as a typist. Lilli starts working on plutonium chemistry, and later transfers to the explosives group when concerns arise about potential reproductive damage from exposure to plutonium. She signs a petition advocating demonstrating the atomic bomb's destructive power as a warning to Imperial Japan, rather than dropping it on civilian populations there. Lilli went on to found Higher Education Resource Services, a nonprofit working to advance females pursing higher education and wrote/edited three books on women in science.
      \item[] \imp{1944} -\ Toni Oppenheimer, daughter of J. Robert Oppenheimer, Director of the Los Alamos National Laboratory, and Katherine ``Kitty'' Oppenheimer, is born at Los Alamos. Security requires that Toni's birth certificate list the anonymous ``P.O. Box 1663'' as her birthplace. She is one of many children born at the Manhattan Project sites, and is on site until age three when her father becomes the Director of the Institute for Advanced Study in Princeton.
      \item[] \imp{1945} -\ Floy Agnes ``Aggie'' Lee graduates from the University of New Mexico and moves to Los Alamos to join the hematology laboratory. Aggie collects blood from the researchers, and analyzes the blood cell information. She frequently plays Enrico Fermi in tennis after work, without knowing who he is, as all employees are referred to by employee number instead of by name for secrecy. Aggie's parents are Pueblo Indian and White, and she is one of the few Native Americans working on the Manhattan Project. Later, she works at Argonne National Laboratory and earns her Ph.D., all while being a single mom.
      \item[] \imp{July 1945} -\ A Swarthmore graduate, Frances Dunne, is recruited from her job at Kirtland Air Force Base to begin work at Los Alamos with the Explosives Assembly Group. She is the only woman on the team, and part of the assembly crew for the Trinity test, the world's first nuclear explosion, which explodes on July 16, 1945.
      \item[] \imp{1966} -\ Lise Meitner is the first woman to receive the Enrico Fermi Award for her pioneering research in naturally occurring radioactivity and extensive experimental studies leading to the discovery of fission. The Enrico Fermi Award is one of the oldest and most prestigious science and technology honors bestowed by the federal government. 
    \end{itemize}
    \item \red{X} \textbf{Prominent and deserving women who were left out of the Nobel Prize and why. As well as relationships seen among those who have won the prize.}
    \begin{itemize}
      \item
    \end{itemize}
    \item \red{X} \textbf{The Civil Rights Act of 1964 and gender role.}
    \begin{itemize}
      \item 
    \end{itemize}
    \item \red{X} \textbf{Feminism and the women’s liberation movement -\ how the perceptions of women's role in science changed.}
    \begin{itemize}
      \item
    \end{itemize}
    \item \textbf{Topics covered in Why so Few?}
    \begin{itemize} 
      \item \href{https://www.cbs.com/shows/60_minutes/video/YPhAPrmoPNXU6EEPdRzkbcRP7oM_1RuR/grace-hopper-she-taught-computers-to-talk/}{\warn{Grace Hopper Segment on 60 minutes, (In \link{Notable Women})}}
      \item \red{X} \href{https://www.youtube.com/watch?v=01ZrwW9C5z0}{\warn{Cathy Wolf, Human-Comp Inter. Pioneer, (In \link{Notable Women})}}
      \item \red{X} \imp{Madame Curie Complex: The Hidden History of Women in Science}
      \item \red{X} \imp{Women's Contribution to ENIAC} -\ \warn{/mse/txt/eniac.txt}
      \item \dne{Women's contributions to the National Bureau of Standards.}
      \item \red{X} \href{https://www.npr.org/sections/money/2014/10/21/357629765/when-women-stopped-coding}{\warn{When women stopped coding}}
      \item \red{X} \href{https://www.theverge.com/2017/8/16/16153740/tech-diversity-problem-science-history-explainer-inequality}{\warn{Science doesn't explain tech's diversity problem, history does}}
      \item \red{X} \href{https://www.aauw.org/resources/research/the-stem-gap/}{\warn{Why so few? Women in Science, Tech, Engn. and Math}} % /mse/why.pdf
      \item \red{X} \href{https://www.youtube.com/watch?v=QgUVrzkQgds&feature=youtu.be}{\warn{Calculating Ada - the countess of computing}}
      \item \href{https://www.theguardian.com/science/sifting-the-evidence/2015/oct/13/why-ada-lovelace-day-matters}{\warn{Why Ada Lovelace Day Matters, (In \link{Notable Women})}}
      \item \href{https://web.archive.org/web/20170312212947/http:/www.doublexscience.org/the-finkbeiner-test/}{\warn{The Finkbeiner Test}} \imp{\&} \href{https://publiceditor.blogs.nytimes.com/2013/04/01/gender-questions-arise-in-obituary-of-rocket-scientist-and-her-beef-stroganoff/}{\warn{Gender Questions Arise in Obituary}}: Women truly face adversity after adversity in their path to success in any field and especially in STEM fields. Not only are their works often ignored by male colleagues, but in receiving praise in a male-dominated industry, they are faced with qualifiers such as ``being good as a female'' rather than being viewed as a professional in their industry in general. Margaret Sullivan explores how obituaries should commemorate female professionals in her article ``Gender Questions Arise in Obituary of Rocket Scientist and Her Beef Stroganoff.'' In the first paragraph of Yvonne Brill’s obituary in Times, was remembered as a woman who took off eight years from work to raise three children and ``made a mean beef stroganoff.'' Out of frustration on how brilliant women scientists are remembered, Ann Finkbeiner sought to write about articles that celebrated women scientists for their work rather than focusing on their gender-this became known as The Finkbeiner Test. When notable professional women are remembered for their domestic responsibilities first and foremost, it becomes an insult to the work that they’ve spent their life long career developing. However, without mentioning the fact that a professional female scientist is a woman, undermines the environmental hardships she must have faced. Yvonne Brill was a brilliant rocket scientist who in the early 1970s invented a propulsion system to keep communications satellites from slipping out of their orbit. Her very accomplishments in the scientific field is the main reason for the display of her obituary as the lead in Times. Yet to have the first paragraph of her obituary reference to her domestic duties is an insult to her professional work that has made her notable to be written about in the papers. The Finkbeiner Test is a great guideline to follow to remember women professionals, in which focusing on their work as an expert in their field as the most notable aspect of their success. However, articles should in fact celebrate when an accomplished professional happens to be a woman to shine the spotlight on how their achievement is even more notable than a similarly accomplished male counterpart because of all the prejudices we know women face in their road to success.
    \end{itemize}
    \item \red{X} \textbf{How the number of women in STEM has changed through the years. Which fields have seen growth/drop?}
    \begin{itemize}
      \item 
    \end{itemize}
    \item \red{X} \textbf{Increasing African American Women to Engineering.}
    \begin{itemize}
      \item \href{https://www.nsbe.org/getattachment/News-Media/NSBE-News/ignored-potential/NSBE_IgnoredPotential_Whitepaper_TXT-FINAL.PDF.aspx}{\warn{Ignored Potential}} \imp{\&} \warn{/mse/pdf/africanWomen.pdf}
    \end{itemize}
  \end{enumerate}

  \clearpage
  \begin{center}
    \large\link{Other Links}
  \end{center}
  \begin{enumerate}
    \item \red{X} \href{https://www.youtube.com/watch?v=p12jUNsx5e0&feature=youtu.be}{\warn{Women in Aviation}}
    \item \red{X} \href{https://www.youtube.com/watch?v=FKaOGuzhbM8#action=share}{\warn{Solving the Equation}}
    \item \red{X} \href{https://ncses.nsf.gov/pubs/nsf19304/digest}{\warn{National Science foundation}}
    \item \red{X} \href{https://www.usatoday.com/story/news/nation/2019/10/18/nasa-astronauts-international-space-station-first-all-female-spacewalk/4020056002/}{\warn{2 astronauts made history in nasa first all female spacewalk}}
  \end{enumerate}
  

  \clearpage
  \begin{center}
    \large\link{Notable Women}
  \end{center}
  \begin{enumerate}
    \item \imp{Elizabeth Rona} -\ Elizabeth Rona was born on March 20, 1890, in Budapest to Ida Mahler and Samuel Rona. Her father was a doctor, and she wanted to become a doctor, too. But her father, fearing the work would be difficult for a woman, encouraged her to study chemistry instead. After earning her PhD at the University of Budapest, she worked as a researcher with the radiochemist Kasimir Fajans at the University of Karlsruhe (now the Karlsruhe Institute of Technology) in Germany. She returned to Hungary and worked with George von Hevesy, a chemist whose experiments used radioactive versions of elements as tracers to explore chemical reactions. Rona and von Hevesy tracked the diffusion of radioactive tracers in various materials to see how quickly atoms of a substance drifted from one area to another. With that information, it would be possible to calculate an atom's size, which could then be used to help explain its behavior. Von Hevesy and Rona's findings were important for other scientists, but the tracers, for which von Hevesy eventually won the Nobel Prize, turned out to have even broader uses:\ For decades now, doctors have injected radioactive tracers into patients to aid in diagnosing conditions like cancer and heart disease. Rona was later invited to work with the radiochemist Otto Hahn in Berlin. She arrived at the Kaiser Wilhelm Institute in 1921 to do research alongside scientific giants like Lise Meitner, who along with Hahn would contribute to the discovery of nuclear fission, the reaction at the heart of the atom bomb. Rona set to work isolating a potential new element called ionium. The substance turned out to be an isotope of the naturally occurring radioactive metal thorium, which had already been well studied. But thorium would play an important role in Rona's later career. Trying to escape the hardships of post-World War I Berlin, Rona briefly worked as an industrial chemist at a textile plant in Hungary, where she developed a method to turn flax into a burlap-like material. When Stefan Meyer, the director of the Radium Institute in Vienna, offered her a research job in his lab, she moved again. Scientists knew that exposing atoms to a radioactive element like radium would trigger reactions, revealing details of their internal structure and physics. But radium was rare and costly, with tiny quantities hoarded and chivvied between nations for experiments. Researchers soon learned that polonium could be used in these experiments instead. Rona was one of those who learned to prepare the polonium, which made her in demand by high-profile labs. Rona traveled to Paris to work on making concentrated polonium sources alongside Irene Joliot-Curie, Marie Curie's daughter. Rona's polonium went on to be used in numerous experiments. In the years following her Paris sojourn, Rona started to explore the phenomenon of radioactivity in seawater. Each summer, in an island laboratory in Sweden, she measured natural radioactivity in the ocean. She soon realized that there were important differences between the radioactivity of what was floating in the water and what settled on the seabed. Again, politics interrupted her research. The rise of the Nazis forced her to flee Europe for the United States. To aid in the war effort, she gave the United States her methods for concentrating polonium, which were used in the project of building an atomic bomb. She later became a teacher at the Oak Ridge Institute of Nuclear Studies in Oak Ridge, Tenn., and a professor of oceanography at the University of Miami. She continued her work on seawater, analyzing its uranium content. Uranium levels were constant in oceans across the world, she found. Thorium, however, sinks to the sea floor. That discovery had important implications:\ Anything growing in seawater, like a coral reef, will have uranium but no thorium in it —\ at least at first. Over the course of eons, uranium gradually breaks down into thorium. It happens at such a predictable rate that scientists can deduce the age of an ancient reef from the amount of thorium in it. Furthermore, the fact that thorium builds up on the sea floor can also be used to date deep sea sediment cores. This set of radioactive clocks, which draw on observations that Rona made over the course of her peripatetic life, have been used around the world for more than 50 years to construct pictures of past tectonic events and measure how sea level has changed. Despite the many dangers of her time — from radiation, revolutions and wars — Rona worked on radioactivity for nearly six decades, contributing to one of the most important chapters of its study. She died on July 27, 1981. She was 91.
    \item \dne{Leona Libby} -\
    \item \imp{Cecilia Payne-Gaposchkin} -\ \red{X} \href{https://www.aip.org/history-programs/niels-bohr-library/oral-histories/4620}{\warn{Cecilia Payne-Gaposchkin}}
    \item \imp{Maria Goeppert Mayer} -\ \red{X} \warn{/mse/pdf/rosalindFranklin.pdf}
    \item \imp{Chien-Shiung Wu} -\ Known as the first lady of physics and queen of nuclear research. She studied at Berkeley after getting her degree in China. She then accepted a position at Princeton as the first female instructor. In 1944, she joined the Manhattan Project at Columbia University during WWII that produced the first nuclear weapon, she helped develop the process for separating uranium into the u-235 and u-238 isotopes by gaseous diffusion producing large quantities of uranium as as fuel for atomic bombs. After leaving the Manhattan Project, she spent the rest of her life in the physics department at Columbia University. Meanwhile, she was approached by Sung Da Lee and Chen Ming Young, two theoretical physicists who grew to question a hypothetical law and elementary particle physics, the law of conservation of paradis. She then conducted an experiment that disproved the law of parity, which led a Nobel Prize for the male physicists in 1957, but Wu was excluded. In 1958, she was the first women to earn the Research Corporation award and the 7th women elected to the National Academy of Sciences, she was also the first women to serve as the president of the American Physical Society. 
    \item \imp{Maria Mayer} -\ \red{X} \warn{/mse/pdf/rosalindFranklin.pdf}
    \item \dne{Gerty Cori} -\
    \item \imp{Dorothy Crowfoot Hodgkin} -\ helped with the advancements of x-rays crystallography which is a tool used to see the arrangement of atoms and molecules. She was born in Cairo Egypt in 1910 to archaeologists, she spent her childhood in Egypt in England but spent much time away from her parents because of WWI.\ She studied chemistry at Oxford, then phd for philosophy from University of Cambridge and it was here where she became interested in x-ray crystallography. She earned the Nobel Prize in Chemistry in 1964 because she was able to decipher the structures of insulin, penicillin and vitamin b12 which further helped in diabetes and infectious diseases.
    \item \imp{Rosalind Franklin} -\ \red{X} \warn{/mse/pdf/rosalindFranklin.pdf}
    \item \imp{Rosalyn Sussman Yalow} -\ \red{X} \warn{/mse/pdf/rosalynSussman.pdf}
    \item \dne{Fay Ajzenberg-Selove's} -\
    \item \imp{Barbara McClintock} -\ \red{X} \warn{/mse/pdf/rosalindFranklin.pdf}
    \item \imp{Jane Goodall} -\ \red{X} \warn{/mse/pdf/janeDiane.pdf}
    \item \imp{Diane Fossey} -\ \red{X} \warn{/mse/pdf/janeDiane.pdf}
    \item \imp{Birute Galdikas} -\ \red{X} \warn{/mse/pdf/janeDiane.pdf}
    \item \dne{Louis Leakey} -\
    \item \imp{Margarett Hamilton} -\ In 1958, received her undergraduate degree in mathematics and was married shortly after. She put graduate school on hold, and got a job at MIT as a programmer to support her husband while he pursued his law degree at Harvard. She learned to predict weather and detect enemy planes which got her a job at NASA.\ She discovered a flaw on rocket because her daughter was playing in the simulator and pressed a button, so she programmed the software to correct for this error just in case the astronauts would do the same. July 20th 1969, Apollo 11 was able to land on the moon thanks to her software program. Software programming is popularized because of her. \href{https://www.wired.com/2015/10/margaret-hamilton-nasa-apollo/}{\warn{Her code got humans to the moon and invented software itself}}
    \item \imp{Sally Ride} -\ Dr.\ Sally Ride, the first American woman to fly in space, cofounded Sally Ride Science in 2001 along with Dr.\ Tam O'Shaughnessy,\ Dr.\ Karen Flammer,\ and two like minded friends to inspire young people in STEM and to promote STEM literacy. Their goal is to motivate more students especially girls and minorities to stick with STEM as they go through school. Sally served as CEO of the company until her death on July 23, 2012 after a 17 month battle with pancreatic cancer. Sally was finishing her phd in physics at Stanford University in 1977 when she saw an article in the student newspaper saying that NASA was seeking astronaut candidates and that for the first time women could apply. When she blasted off aboard the space shuttle Challenger on June 18, 1983 she became the first American woman and at 32 the youngest American in space. Sally's historic flight made her a symbol of the ability of women to break barriers and a hero to generations of adventurous young girls. She flew on Challenger again in 1984 and later was the only person to serve on both panels investigating the nation's space shuttle disasters the Challenger explosion in 1986 and the breakup of the shuttle Columbia on reentry in 2003. After retiring from NASA, Sally became a physics professor at the University of California, San Diego. She also joined with O'Shaughnessy to write award winning science books for children. Sally used her high profile to champion a cause she cared about passionately igniting students enthusiasm for science and picking their interest in careers in STEM.\ She cofounded Sally Ride Science to help achieve that goal. In 2015, UC San Diego acquired Sally Ride Science and created a new nonprofit organization, Sally Ride Science at UC San Diego. O’Shaughnessy is the executive director and Flammer is the director of education.
    \begin{itemize}
      \item \sub{GROWING UP}:\ Sally was born on May 26, 1951, in Los Angeles, California, and she spent her childhood there. As a young girl, Sally was fascinated by science. She credited her parents with encouraging her interests. Sally grew up playing with a chemistry set and a telescope. She also grew up playing sports. She competed in national junior tennis tournaments and was good enough to win a tennis scholarship to Westlake School for Girls in Los Angeles.
      \item \sub{BECOMING AN ASTRONAUT}:\ In 1977, Sally already had degrees in physics and English from Stanford University and was about to finish her phd in physics when she saw an article in the Stanford student newspaper saying that NASA was looking for astronauts. Up until then, most astronauts had been military pilots and they all had been male. But now NASA was looking for scientists and engineers and was allowing women to apply. Sally immediately sent in her application along with 8,000 other people. From that group, 35 new astronauts, including six women, were chosen to join the astronaut corps. NASA selected Sally as an astronaut candidate in January 1978.
      \item \sub{ASTRONAUT TRAINING}:\ Sally's astronaut training included parachute jumping, water survival, weightlessness, radio communications, and navigation. She enjoyed flight training so much that flying became one of her hobbies. During the second and third flights of the space shuttle Columbia, she worked on the ground as a communications officer, relaying messages from mission control to the shuttle crews. She was part of the team that developed the robot arm used by shuttle crews to deploy and retrieve satellites.
      \item \sub{SPACE MISSIONS}:\ In August 1979, after a yearlong training and evaluation period, Sally became eligible for assignment as an astronaut on a space shuttle flight crew. Then on April 19, 1982 NASA's flight crew director summoned Sally into his office and told her she had been selected as a mission specialist for mission STS-7 aboard the shuttle Challenger. When Challenger blasted off from Kennedy Space Center, Florida on June 18, 1983 Sally soared into history as the first American woman in space. Accompanying Sally aboard Challenger were Captain Robert L.\ Crippen, the spacecraft commander Captain Fredrick H.\ Hauck, the pilot and fellow mission specialists Colonel John M.\ Fabian and Dr.\ Norman E.\ Thagard.\ This was the second flight for the orbiter Challenger and the first mission with a five person crew. During the mission, the crew deployed satellites for Canada (ANIK C-2) and Indonesia (PALAPA B-1) operated the Canadian built robot arm to perform the first deployment and retrieval with the Shuttle Pallet Satellite (SPAS-01) conducted the first formation flying of the shuttle with a free flying satellite (SPAS-01) carried and operated the first U.S./German cooperative materials science payload (OSTA-2) and operated the Continuous Flow Electrophoresis System (CFES) and the Monodisperse Latex Reactor (MLR) experiments. The crew also activated seven Getaway Specials small experiments sent into space by private individuals or groups. The mission lasted 147 hours before Challenger landed on a lake bed runway at Edwards Air Force Base, California, on June 24. Sally's second flight was the 13th shuttle flight, STS 41-G, which launched from Kennedy Space Center on October 5, 1984. The crew of seven the largest to date for a shuttle mission included Crippen as commander, Captain Jon A.\ McBride as the pilot, fellow mission specialists Dr.\ Kathryn D.\ Sullivan and Commander David C.\ Leestma, and two payloads specialists, Commander Marc Garneau and Paul Scully Power. During the 8 day mission, the crew deployed the Earth Radiation Budget Satellite and conducted scientific observations of the Earth with the OSTS-3 pallet and Large Format Camera, as well as demonstrating potential satellite refueling with an EVA and associated hydrazine transfer. After 197 hours in flight, Challenger landed at Kennedy Space Center on October 13. In June 1985, Sally was assigned to the crew of STS 61-M, but mission training was halted in January 1986 after the Challenger exploded shortly after takeoff, killing all seven crew members. Sally served on the presidential commission investigating the tragedy. After the investigation was completed, she was assigned to NASA headquarters as special assistant to the administrator for long range and strategic planning. There she wrote an influential report entitled ``Leadership and America's Future in Space'' and became the first director of NASA's Office of Exploration.
      \item \sub{BEYOND SPACE}:\ Sally retired from NASA in 1987. She became a science fellow at the Center for International Security and Arms Control at Stanford University. In 1989, Sally joined the faculty at the University of California, San Diego as a professor of physics and director of the California Space Institute. Long an advocate for improved science education, Sally co-wrote seven science books for children—To Space and Back (with Sue Okie) and Voyager The Third Planet The Mystery of Mars; Exploring Our Solar System Mission Planet Earth; and Mission Save the Planet (all with Tam O’Shaughnessy). Sally also initiated and directed NASA-funded education programs designed to fuel middle school students’ fascination with science, including Sally Ride EarthKAM and GRAIL MoonKAM.\ Sally was a member of the President's Committee of Advisors on Science and Technology and the National Research Council's Space Studies Board, and she served on the boards of the Congressional Office of Technology Assessment, the Carnegie Institution of Washington, and the NCAA Foundation. Sally was a fellow of the American Physical Society and a member of the Pacific Council on International Policy, and she served on the boards of the Aerospace Corporation and the California Institute of Technology. Sally received numerous honors and awards. She was inducted into the National Women's Hall of Fame, the California Hall of Fame, the Aviation Hall of Fame, and the Astronaut Hall of Fame, and she received the Jefferson Award for Public Service, the von Braun Award, the Lindbergh Eagle, and the NCAA's Theodore Roosevelt Award. She was twice awarded the NASA Space Flight Medal. In 2012 Sally was honored with the National Space Grant Distinguished Service Award. In November 2013, Sally was posthumously awarded the Presidential Medal of Freedom in a White House ceremony where O'Shaughnessy her life partner and co-founder of Sally Ride Science accepted the medal on her behalf. Also attending the ceremony were Sally's mother, Joyce Ride, and sister, Bear Ride. Other 2013 medal recipients included President Bill Clinton, Gloria Steinem, and Oprah Winfrey. Sally's life and work continue to be recognized with other posthumous honors. Sally and another pioneering astronaut, Neil Armstrong, received The Space Foundation's 2013 General James E.\ Hill Lifetime Space Achievement Award for their contributions to space exploration. Also in 2013, the Stanford School of Engineering named Sally a Stanford Engineering Hero, an honor bestowed on Stanford scientists who have benefited humanity through engineering and science. In 2014, Women in Aviation, International (WAI) inducted Sally into its International Pioneer Hall of Fame.\ Ray Mabus, then secretary of the Navy, announced in 2013 that a state of the art research vessel would be named in Sally's honor. The 238 foot R/V Sally Ride, owned by the Navy and operated by Scripps Institution of Oceanography at UC San Diego, was commissioned in 2016. The Sally Ride is a Neil Armstrong class auxiliary general oceanographic research (AGOR) ship. It is the first Navy research ship named for a woman. In 2018, the U.S.\ Postal Service unveiled a Forever stamp honoring Sally. And in 2019, Stanford University renamed a student housing complex Sally Ride House.
    \end{itemize}
    \item \imp{Rachel Carson} -\ A marine biologist and nature writer, Rachel Carson catalyzed the global environmental movement with her 1962 book Silent Spring. Outlining the dangers of chemical pesticides, the book led to a nationwide ban on DDT and other pesticides and sparked the movement that ultimately led to the creation of the US Environmental Protection Agency (EPA). Born on May 27, 1907 on a farm in Springdale, Pennsylvania, Carson was the youngest of Robert and Maria McLean Carson’s three children. She developed a love of nature from her mother, and Carson became a published writer for children’s magazines by age 10. She attended the Pennsylvania College for Women (now Chatham University), graduating magna cum laude in 1929. She next studied at the oceanographic institute at Woods Hole, Massachusetts and at Johns Hopkins University, where she received a master’s degree in zoology in 1932. Strained family finances forced her to forego pursuit of a doctorate and help support her mother and, later, two orphaned nieces. After outscoring all other applicants on the civil service exam, in 1936 Carson became the second woman hired by the US Bureau of Fisheries. She remained there for 15 years, writing brochures and other materials for the public. She was promoted to Editor-in-Chief of all publications for the US Fish and Wildlife Service. Meanwhile, she wrote several popular books about aquatic life, among them Under the Sea Wind (1941) and The Sea Around Us (1951). The latter was serialized in the New Yorker and sold well worldwide. She won a National Book Award, a national science writing-prize and a Guggenheim grant, which, with the book’s sales, enabled her to move to Southport Island, Maine in 1953 to concentrate on writing. In 1955, she published The Edge of the Sea, another popular seller. She also began a relationship with Dorothy Freeman, a married summer resident. Though much of their correspondence was destroyed shortly before Carson’s death, the rest was published by Freeman’s granddaughter in 1995 as Always, Rachel: The Letters of Rachel Carson and Dorothy Freeman, 1952–1964: An Intimate Portrait of a Remarkable Friendship. After a niece died in early 1957, Carson adopted her son and relocated to Silver Spring, Maryland, to care for her aging mother. A letter from a friend in Duxbury, Massachusetts about the loss of bird life after pesticide spraying inspired Carson to write Silent Spring. The book primarily focuses on pesticides' effects on ecosystems, but four chapters detail their impact on humans, including cancer. She also accused the chemical industry of spreading misinformation and public officials of accepting industry claims uncritically. Chemical companies sought to discredit her as a Communist or hysterical woman. Many pulled their ads from the CBS Reports TV special on April 3, 1963, entitled ``The Silent Spring of Rachel Carson.'' Still, roughly 15 million viewers tuned in, and that, combined with President John F. Kennedy’s Science Advisory Committee Report—which validated Carson’s research—made pesticides a major public issue. Carson received medals from the National Audubon Society and the American Geographical Society, and induction into the American Academy of Arts and Letters.  Seriously ill with breast cancer, Carson died two years after her book’s publication. In 1980, she was posthumously awarded the Presidential Medal of Freedom. Her homes are considered national historic landmarks, and various awards bear her name.
    \item \imp{Hedy Lamarr} -\ Hedy Lamarr was an Austrian-American actress and inventor who pioneered the technology that would one day form the basis for today's WiFi, GPS, and Bluetooth communication systems. As a natural beauty seen widely on the big screen in films like Samson and Delilah and White Cargo, society has long ignored her inventive genius. Lamarr was originally Hedwig Eva Kiesler, born in Vienna, Austria on November 9th, 1914 into a well-to-do Jewish family. An only child, Lamarr received a great deal of attention from her father, a bank director and curious man, who inspired her to look at the world with open eyes. He would often take her for long walks where he would discuss the inner-workings of different machines, like the printing press or street cars. These conversations guided Lamarr’s thinking and at only 5 years of age, she could be found taking apart and reassembling her music box to understand how the machine operated. Meanwhile, Lamarr’s mother was a concert pianist and introduced her to the arts, placing her in both ballet and piano lessons from a young age. Lamarr's brilliant mind was ignored, and her beauty took center stage when she was discovered by director Max Reinhardt at age 16. She studied acting with Reinhardt in Berlin and was in her first small film role by 1930, in a German film called Geld auf der Strate ``Money on the Street''. However, it wasn't until 1932 that Lamarr gained name recognition as an actress for her role in the controversial film, Ecstasy. While in London, Lamarr's luck took a turn when she was introduced to Louis B. Mayer, of the famed MGM Studios. With this meeting, she secured her ticket to Hollywood where she mystified American audiences with her grace, beauty, and accent. In Hollywood, Lamarr was introduced to a variety of quirky real-life characters, such as businessman and pilot Howard Hughes. Lamarr dated Hughes but was most notably interested with his desire for innovation. Her scientific mind had been bottled-up by Hollywood but Hughes helped to fuel the innovator in Lamarr, giving her a small set of equipment to use in her trailer on set. While she had an inventing table set up in her house, the small set allowed Lamarr to work on inventions between takes. Hughes took her to his airplane factories, showed her how the planes were built, and introduced her to the scientists behind process. Lamarr was inspired to innovate as Hughes wanted to create faster planes that could be sold to the US military. She bought a book of fish and a book of birds and looked at the fastest of each kind. She combined the fins of the fastest fish and the wings of the fastest bird to sketch a new wing design for Hughes' planes. She went on to create an upgraded stoplight and a tablet that dissolved in water to make a soda similar to CocaCola. However, her most significant invention was engineered as the United States geared up to enter World War II.\ In 1940 Lamarr met George Antheil at a dinner party. The two came up with an extraordinary new communication system used with the intention of guiding torpedoes to their targets in war. The system involved the use of ``frequency hopping'' amongst radio waves, with both transmitter and receiver hopping to new frequencies together. Doing so prevented the interception of the radio waves, thereby allowing the torpedo to find its intended target. After its creation, Lamarr and Antheil sought a patent and military support for the invention. While awarded U.S. Patent No.\ 2,292,387 in August of 1942, the Navy decided against the implementation of the new system. The rejection led Lamarr to instead support the war efforts with her celebrity by selling war bonds. Happy in her adopted country, she became an American citizen in April 1953. Meanwhile, Lamarr's patent expired before she ever saw a penny from it. While she continued to accumulate credits in films until 1958, her inventive genius was yet to be recognized by the public. It wasn't until Lamarr's later years that she received any awards for her invention. The Electronic Frontier Foundation jointly awarded Lamarr and Antheil with their Pioneer Award in 1997. Lamarr also became the first woman to receive the Invention Convention's Bulbie Gnass Spirit of Achievement Award. Although she died in 2000, Lamarr was inducted into the National Inventors Hall of Fame for the development of her frequency hopping technology in 2014. Such achievement has led Lamarr to be dubbed ``the mother of Wi-Fi'' and other wireless communications like GPS and Bluetooth. 
    \item \imp{Grace Hopper} -\ \red{X} She was sent to Harvard to help program the first computer, Mark I. Also a very proud navy.
    \item \imp{Ada Lovelace} -\ Ada Lovelace Day was founded in 2009 by Suw Charman-Anderson, and part of her reason for doing this was a worry that women in tech were invisible. The idea was a positive one -\ rather than highlighting the problem, highlight the unseen women and shout from the rooftops about all the amazing things they’ve achieved. Ada Lovelace was an obvious choice of mascot for such an endeavour. Lovelace was Lord Byron’s daughter, though she didn’t know her father very well. She was schooled in maths and science, unlike the majority of girls at the time she was growing up. Her social circle included Charles Babbage, and her grasp of the potential for his Analytical Engine has led her to be hailed as the first computer programmer. We know that women are underrepresented in science, and because of initiatives like Ada Lovelace Day, and organizations like ScienceGirl, there are more and more people attempting to do something about the unconscious biases that may well be responsible for the discrepancy. But women are not the only group underrepresented. Ada herself was a wealthy and highly educated woman. And these days academia still has more than its fair share of white, middle class employees. Perhaps as well as Ada Lovelace Day we need a day championing scientists, engineers and mathematicians who don’t fit this mould, in the hope that increasing the visibility of these people will encourage more diversity in future. \href{https://www.theguardian.com/science/sifting-the-evidence/2015/oct/13/why-ada-lovelace-day-matters}{\warn{Why Ada Lovelace Day Matters}}
    \item \imp{Katherine Johnson} -\ Katherine G. Johnson is a pioneer in American space history. A NASA mathematician, Johnson's computations have influenced every major space program from Mercury through the Shuttle. She even calculated the flight path for the first American mission space. Born in 1918 in West Virginia, Johnson was a talented student who entered college at only 15 years old. At West Virginia State University, W.W. Schiefflin Clayor, the third African American to earn a PhD. in mathematics, recognized Johnson's abilities and motivated her to take advanced math. Johnson would go on to earn a graduate degree in mathematics. In 1953, Johnson was contracted as a research mathematician at the Langley Research Center with the National Advisory Committee for Aeronautics, the agency that preceded NASA.\ She worked in a pool of women performing math calculations until she was temporarily assigned to help the all male flight research team and wound up staying there. Johnson's specialty was calculating the trajectories for space shots which determined the timing for launches, including the Mercury mission and Apollo 11, the mission to the moon.
    \item \imp{Paula Hammond} -\ Childhood curiosity in northwest Detroit:\ father had phd in biochemistry and mother had masters in nursing, her father would interest her into chemistry. It was in high school where she discovered chemical engineering. She fell in love with MIT on first site where she earned her degree in chemical engineering and also where she found her husband John Hammond a mechanical engineering student. Together they ended up working at Motorola as the first African-American process engineers. Later, her and her husband pursed to further their education. She got her master's in chemical engineering at Georgia Institute of Technology. Then she returned to MIT for her phd, but she also had daughter and separated, being a single mother. Her daughter became a transgender and also her son is studying psychology at Northeastern. She created polymers for revolutionary drug delivery systems. The first of her three major research areas. She developed nano-particles for cancer. Nano technology for soldiers. In 2009, Paula presented her research on polymer batteries to Obama, which informed his speech on clean energy. The sponges Paula designed for the battlefield she hopes to soon develope for use by paramedics and hospitals. Pursuing new ways to deliver RNA in the treatment of cancer and infectious disease.
    \item \imp{Lydia Villa-Komaroff} -\ The early influences of her mother and uncle set Lydia Villa-Komaroff on the path to becoming a scientist. Along the way, numerous other mentors influenced her decision to obtain her PhD in biology and also taught her important life lessons. As a post-doctoral fellow, Lydia Villa-Komaroff was a key member of the team that showed for the first time that bacteria could be induced to produce insulin. She spent the next twenty years researching growth factors and development before moving to science administration. Currently, she is Chief Scientific Officer at CytonomeST.\ Amongst her many accomplishments, she counts being a founding member of the Society for the Advancement of Chicanos and Native Americans in Science. \warn{/mse/pdf/lydia.pdf}
    \item \imp{Catherine Wolf} -\ \red{X} A pioneering psychologist in the field of human-computer interaction, has been living with amyotrophic lateral sclerosis (ALS, or Lou Gehrig's disease) for 20 years. She invented first gesture-based system for collaboration called We Met.
    \item \imp{Mildred Dresselhaus} -\ Mildred Dresselhaus passed away at the age of 86 on February 20, 2017. Prior to her passing, she spoke with MAKERS about her career as a scientist. Known as the ``queen of carbon science'', Mildred Dresselhaus was a pioneer in nanoscience and a strong advocate of bringing more women into STEM.\ Born in New York on November 11, 1930, Dresselhaus originally planned to become a teacher. After receiving her Bachelor's degree from Hunter College in 1951 and later continuing her studies at Cambridge University as a Fulbright Fellow, she was encouraged by Nobel Prize winner Rosalyn Yalow to pursue physics. This change of direction led her to earn her Master's degree from Radcliffe College and later her PhD from the University of Chicago, where she was mentored by the world-renown physicist Enrico Fermi. In 1960, Dresselhaus began working at the Massachusetts Institute of Technology, where she stayed for 57 years. During her time there, she worked in different positions at the Solid State Division of Lincoln Laboratory and the Department of Electrical Engineering, eventually becoming a member of the electrical engineering faculty in 1968. There, Dresselhaus became the first female professor at MIT to hold a full, tenured position and helped organize the school's first Women's Forum. Outside of MIT, Dresselhaus served as the director of the Office of Science at the U.S. Department of Energy; president of the American Physical Society and American Association for the Advancement of Science; chair of the governing board of the American Institute of Physics; and treasurer of the National Academy of Sciences. Throughout her successful career, Dresselhaus became a woman of many firsts, receiving the first solo Kavli Prize and becoming the first woman to win the National Medal of Science in Engineering. She is also the winner of both the Presidential Medal of Freedom and National Medal of Honor and the co-author of eight books and approximately 1,700 papers. \href{https://spectrum.ieee.org/geek-life/profiles/mildred-dresselhaus-the-queen-of-carbon}{\warn{Mildred Dresselhaus}}
    \item \imp{Marie Tharp} -\ Oceanic cartographer Marie Tharp helped prove the theory of continental drift with her detailed maps of the ocean floor. She landed a position at Columbia University in 1947, as an assistant to Bruce Heezen who was collecting depth measurements across the Atlantic Ocean. During expedition, they used echo sounding collect depth data which involves sending out high-frequency sounds/pings which records time delay returning back. Then these data were plotted to build a profile of the terrain below. During these times, women were not allowed to go on these expeditions, so she remained at the university to process the data converting rows of X measurement into detailed profile of the ocean floor. Initially, the ocean floor was thought to be flat, but she discovered the emergence of a long v-shaped cleft which supported the vengan as continental draft theory. Heezen was skeptical of Tharps' idea of the continental drift because he thought it was just ``girl talk''. Another student, Howard Foster at the same time was plotting earthquakes in the same region of the Atlantic Tharp was plotting her continental drifts. After this discovery, Heezen had no choice but to take her claims more seriously. Renowned explorer Jacques Cousteau believed this idea was wrong so he offered to go on a expedition to file the ocean floor which in turn proved Tharps' claims to be true. This would soon give Tharp the position of the most outstanding cartographers of the 20th century.
    \item \imp{Cady Coleman} -\ As a chemist and an astronaut, Cady Coleman boasts an impressive list of accolades that orbit around science and space. The former United States Air Force officer has logged nearly 4,500 hours and 180 days in space as a NASA astronaut. She's a veteran of two space shuttle missions, and participated in a six-month tour on the International Space Station. Coleman is also an American chemist with a doctorate in polymer science and engineering from University of Massachusetts Amherst and a Bachelor's in chemistry from Massachusetts Institute of Technology. While completing work for her Ph.D. at UMASS, she joined the U.S. Air Force as Second Lieutenant. She retired from the Air Force in 2009. Coleman later served as a liaison to NASA's newest Commercial Space partners where she helps assemble and integrate supply ship operations aboard the International Space Station. She retired from NASA in 2016.
    \item \imp{Ellen Ochoa} -\ Ellen Ochoa is an inventor, NASA astronaut, and director of Johnson Space Center. In addition to being the first Hispanic-American woman in space, she holds three U.S. patents. Listen to her story on becoming an astronaut and what she did to commemorate her first flight into space. As she was headed to space, she got flags from the National Women's Party that was used early of the 20th century as women were fighting for their right to vote. Then she took pictures and was proud to represent as a women in space.
    \item \imp{Edith Clark} -\ A pioneering female electrical engineer, born in 1883 rural Maryland, one of eight children. She was the first woman to earn an electrical engineering graduate degree from MIT in 1919. She was the first woman employed by General Electric. She was the first woman elected as a fellow and the first woman to present a technical paper to the predecessor society to the famous Institute of Electrical and Electronics Engineers. She was issued a patent for a graphical calculator, which simplified calculations necessary to solve electronic power transmission line problems. She was also inducted into the National Inventors Hall of Fame in 2015, which celebrates the achievements of US inventors and patent holders.
  \end{enumerate}

  \clearpage
  \begin{center}
    \large\link{Latest Statistics on Women in Science and Engineering}
  \end{center}
  \begin{enumerate}
    \item \textbf{The National Science Foundation report's on statistical information about the participation of Women, Minorities, and Persons with Disabilities in Science and Engineering (S\&E) education and employment. This is mandated by the:}
    \begin{itemize}
      \item \imp{The Science and Engineering Equal Opportunities Act (Public Law 96 -\ 516) }
    \end{itemize}
    \item \textbf{Women, persons with disabilities, and underrepresented minority groups representation in Science and Engineering education and employment is smaller than their representation in the U.S. population.}
    \begin{itemize}
      \item \imp{True}
    \end{itemize}
    \item \textbf{Although women have reached parity with men among S\&E bachelor's degree recipients, half of S\&E bachelor's degrees were awarded to women in 2016, they are still underrepresented in S\&E occupations.}
    \begin{itemize}
      \item \imp{True}
    \end{itemize}
    \item \textbf{As an example of the underrepresentation of women in S\&E fields, the share of S\&E research doctorates awarded to women in 2017 was 41\% versus their 51.5\% of the population and 47\% of the labor force.}
    \begin{itemize}
      \item \imp{True}
    \end{itemize}
    \item \textbf{Which of the following is ``correct'' in regard to the latest statistics on educational attainment?}
    \begin{itemize}
      \item \imp{Women's highest degree shares are in psychology and biosciences; the lowest, in computer sciences and engineering.}
      \item \imp{Of all science and engineering (S\&E) degrees awarded in 2016, women earned about half of bachelor's degrees, 44\% of master's degrees, and 41\% of doctorate degrees, about the same as in 2006. However, the proportion of degrees awarded to women in S\&E fields varies across and within broad fields of study.}
      \item \imp{Among students enrolled in graduate school in S\&E fields in 2016, the number (17,630) of female students are larger than those for male students (4\%, 12,970).}
    \end{itemize}
    \item \textbf{Which of the following is correct in regards to women and minority groups?}
    \begin{itemize}
      \item \imp{In 2016, women from underrepresented minority groups earned more than half of the science and engineering (S\&E) degrees awarded to their respective racial and ethnic groups at all degree levels —\ bachelor's, master's, and doctorate.}
    \end{itemize}
    \item \textbf{Which of the following is ``correct'' in regards to employment?}
    \begin{itemize}
      \item \imp{Looking at reasons for not working, women are much more likely than men to report family responsibilities (27\% versus 6\%). Men are much more likely than women to report being retired, perhaps because a majority of older cohorts of scientists and engineers are male}
      \item \imp{Looking at the unemployment rates of scientists and engineers in 2017, the rates for both women (2.9\%) and men (2.6\%) were lower than that of the U.S. labor force (4.4\%), indicating a strong demand for those with S\&E expertise.}
      \item \imp{Women were much more likely than men to report that family responsibilities resulted in their part-time work schedules, whereas men were more likely than woman to report that they were retired from another job.}
    \end{itemize}
    \item \textbf{Which of the following is ``correct'' in regards to Women and underrepresented minorities in terms of occupation?}
    \begin{itemize}
      \item \imp{Female scientists and engineers were more likely than male scientists and engineers to work in a non-S\&E occupation (48\% versus 42\%).}
      \item \imp{Among scientists and engineers working full time in 2017, women generally made less than men in each broad occupational group. Women have lower median salaries than do men in most S\&E occupations.}
      \item \imp{The share of academic doctoral positions held by women has increased, and although underrepresented minorities have also gained ground, their share of these positions remains small.}
    \end{itemize}
    \item \textbf{Which university is the top U. S. baccalaureate institutions of science and engineering doctorate recipients for women?}
    \begin{itemize}
      \item \imp{University of California, Berkley}
    \end{itemize}
    \item \textbf{Doctoral scientists and engineers employed in universities and 4-year colleges according to all academic position were about the same overall for women and men.}
    \begin{itemize}
      \item \imp{False}
    \end{itemize}
  \end{enumerate}

  \clearpage
  \begin{center}
    \large\link{Questions on the Hidden Figures Film}
  \end{center}
  \begin{enumerate}
    \item \textbf{As John Glenn prepared the morning of his first launch into space, the numbers that the IBM computer projected as his point of entry into the atmosphere came into question as they differed from the previous day's numbers. Glenn, just hours before launch, is notified of this and expressed his desire to have the numbers double-checked and requested ``that smart lady'' to refigure the numbers. That smart lady was \underline, who had previously been dismissed from the NASA team since computers like her were no longer needed because of the IBM machine.}
    \begin{itemize}
      \item \imp{Katherine Goble Johnson}
    \end{itemize}
    \item \textbf{At the very onset, it is easy to see the sheer lack of respect shown to the key women in this movie, which were\:}
    \begin{itemize}
      \item \imp{Katherine Goble Johnson, Mary Jackson, and Dorothy Vaughn}
    \end{itemize}
    \item \textbf{Another instance of reverence is shown as \underline{\ldots} is respectfully confirmed as the person who made the numbers work on the IBM machine and is later promoted to supervisor. Later on, she and her colored team are asked to train the white women as to how to work with this computer, demonstrating further acknowledgement and respect. During this Civil Rights era, such a request was unheard of, which further demonstrated the depth of respect given.}
    \begin{itemize}
      \item \imp{Dorothy Vaughn}
    \end{itemize}
    \item \textbf{Katherine Goble Johnson's superior Mr.\ Harrison, asked as to why she is always away from her desk for extensive breaks her reason was because\:}
    \begin{itemize}
      \item \imp{She had to use the ``colored'' bathrooms several blocks away in another building.}
    \end{itemize}
    \item \textbf{The movie illustrates the story of how these three women overcome the obstacles of being both black and female at a time when neither was given the respect deserved.}
    \begin{itemize}
      \item \imp{True}
    \end{itemize}
    \item \textbf{\underline{\ldots} would write up the report with her name on it only to be told by her coworker, Paul, that she was not allowed to include her name in the report because computers did not write reports, even though it was apparent that she had done the work involved in the report. While this destroyed her trust in Paul for some time, as time progressed Paul and others grew to respect \underline{\ldots}, and eventually her name was included in the reports.}
    \begin{itemize}
      \item \imp{Katherine Goble Johnson}
    \end{itemize}
    \item \textbf{All are true in regards to the changes you could see in the way the women were treated except?}
    \begin{itemize}
      \item \imp{No real transformation occurred}
    \end{itemize}
    \item \textbf{The movie Hidden Figures detailed the previous untold story of the contextual experiences of three black women who were brilliant astrophysicists and deeply involved in the early days of NASA in the 1960s.}
    \begin{itemize}
      \item \imp{False}
    \end{itemize}
    \item \textbf{All are examples of shame which were revealed in the movie except?}
    \begin{itemize}
      \item \imp{Having work double checked by male counterparts}
    \end{itemize}
    \item \textbf{As the members of the NASA team continue to work at discovering how to bring the spaceship home safely without burning up in the atmosphere, you see \underline{\ldots}, who successfully petitioned to be allowed to get her degree in an all-white school, was credited and applauded as the person who made this happen.}
    \begin{itemize}
      \item \imp{Mary Jackson}
    \end{itemize}
  \end{enumerate}
\end{document}