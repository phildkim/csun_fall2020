\fancyhead[R]{\textsc{Final Paper}}
\rfoot{\textsc{Page \thepage}}

\par
\begin{itemize}
  \item[] TABLE OF CONTENTS
  \begin{itemize}
    \item INTRODUCTION
    \item SUMMARY
    \item RELATING TO CLASS MATERIALS
    \item WOMEN IN LEADERSHIP ROLES TODAY:\ MY GIRLFRIEND
    \item PRESENT DAY WOMEN SCIENTISTS
    \item BOOK RECOMMENDATION
    \item CONCLUSION
  \end{itemize}
\end{itemize}
\clearpage

\par
\section*{INTRODUCTION}

I’ve personally noticed a movement to empower women in the past 10 years mostly as reflections made in various mediums. It seems that movies, social media, and political realms have made an effort to feature more female characters and leaders who are able to make a difference. Such as in the case of the first Frozen movie which does not constitute a typical Disney movie plot with a damsel in distress saved by a prince charming, but rather Anna realizing that she doesn’t need a man to feel like she can make a difference or as the meaning of her existence. Or in the case of Moana where she’s able to save her island from perishing because she goes on a journey to make Maui restore the heart of Te Fiti or Star Wars’ Last Jedi being a female. A conscious movement to empower women in society is loud and clear especially with most recently Kamala Harris chosen as the Vice President Elect as the first female and first female of color to hold the Vice Presidency in history of the United States.


\par
Although much progress has been made outwardly, the reality is that only 37 females make up the CEOs of the top 500 fortune companies~\cite{fortune}, 26 women elected officials out of 535 in congress~\cite{congress}, and women make on average 81.6 cents for every dollar a man earns~\cite{salary}, and the double standard does not seem to show to slow down as much as we aspire to make progress. One thing that movie makers seem to miss to portray in their characters’ journey is the struggles women in real life have to face to be acknowledged for their achievements compared to their male counterparts. The story of the women in science seems to be the same despite spanning decades. The initial part of getting an education has dramatically improved from the early 1800s, however one thing holds true still today, is that these females from the first generation to modern generation would have been more recognized, studied in science textbooks, and paid more had they been males. 

\par
\section*{SUMMARY}

In her book “Nobel Prize Women in Science: Their Lives, Struggles and Momentous Discoveries,” Sharon Bertsch McGrayne explores the question why so few women have won Nobel Prizes in science, only 10 when more than 500 men have done so. Some were even excluded from the Nobel Prize Award when their work was notable enough to give their husbands or male co-workers credit. McGrayne highlights the struggles these female scientists faced and compares what they share in common. McGrayne divides these scientists into 3 sections: the First Generation Pioneers, the Second Generation, and the New Generation. The First Generation Pioneers include Marie Curie, Lise Meitner, and Emmy Noether. The Second Generation include scientists such as Gerty Cori, Dorothy Hodgkin, and Rosalind Elsie Franklin. And the New Generation include Jocelyn Burnell and Christiane Nüsslein-Volhard.

\par
Women in science have several things in common in their struggles to success. Depending on the time when they lived, their education was severely limited. All of them had the necessary support from male figures; either their husbands, fathers or colleagues, in their work and with the absence of that support, they possibly would not be recognized. Just as McGrayne puts it, ``if a woman formed a long-term scientific partnership with a man, the scientific community assumed that he was the brains of the team and she was the brawn.''~\cite{nobelprize}

\par
The first generation pioneers were brought up in science when women were generally not allowed to attend school, took unpaid jobs and were forced to work in basement labs. Their persistence in the science field was simply for their love of science --\ words like ``pleasure,'' ``joy,'' and ``satisfaction'' were words to describe their work~\cite{nobelprize}. In the case of Marie Curie, she never patented her work for the sake of the advancement of science and radiology to be available to the world. The timing for the first generation pioneers coincided with the first women’s suffrage. This begs the question how many women scientists were under the radar before the women’s suffrage movement that at least allowed some females in some universities. 

\par
\section*{RELATING TO CLASS MATERIALS}

The women included in McGrayne’s book mentions many of the women that we have discussed in class and goes into more depth of their life and struggles. The story unfortunately for all these women are the same and in the case of Emmy Noether we see that the disadvantages these women faced was not a matter of class because Noether came from a prestigious family of mathematicians, and still struggled to be recognized although she made a significant contribution to mathematics second to Albert Einstein, simply because she is a woman. 

\par
Women scientists were not and still have difficulty being recognized for the merit of their work. We hope that through the generations women’s noteworthy discoveries would be recognized and improve in today’s age, 

\clearpage
\par
\section*{WOMEN IN LEADERSHIP ROLES TODAY:\ MY GIRLFRIEND}

The closest example I can relate to the struggles of the women discussed in the book is my girlfriend who accepted an Operations Manager position at a growing wholesale bakery. Although she is not in science, the struggles that she faces on a daily basis resembles the struggles of any woman in a leadership role, in which many of the mundane daily tasks would be much easier if she was a male. She faces many doubts from the managers that she superseded. This power struggle would have occurred in any change in leadership, however, the double standards is very apparent in which she is expected to be a more nurturing leader, she’s expected to do many of the clean up work despite being second only to the owner of the bakery, and has experienced first hand the Matilda Effect in which her ideas are attributed to her male colleagues. 

\par
\section*{PRESENT DAY WOMEN SCIENTISTS}

Although we’ve come a long way from the First Generation Pioneers in which women were largely not admitted to universities, there are still struggles that women scientists face today. Jocelyn Burnell discovered pulsars as a postgraduate student and the questions that reporters were concerned more about her discovery was ``are you taller or shorter than Princess Margaret'' or ``how many boyfriends she had at a time?''~\cite{nobelprize} Despite being the first person to discover radio pulsars, Burnell was not one of the recipients of the Nobel Prize. Instead, the prize recognized her supervisor who was a male. As much progress women have made in the science field, the Matilda Effect seems to follow women scientists wherever part of the world they are from.

\par
\section*{BOOK RECOMMENDATION}

This book compiles the lives, struggles, and achievements of the women who made great contributions to science who time and again were not given enough credit for their life-changing discoveries. Reading about the struggles that they went through gives them even more credit to their achievements. They have become role models to upcoming women in science, and lay out lessons that help advancement for women. I believe any woman struggling in their journey to success would benefit greatly from reading the lessons and struggles the women in science history faced as an encouragement to continue their work and that there is a pathway to success in science despite the existing prejudices.

\par
\section*{CONCLUSION}

The women whose lives are documented by Sharon Bertsch McGrayne have faced immense struggles in the journey to their momentous discoveries, and although the progress in experiences that they’ve had through the generations have generally improved, there is no sight of the struggles disappearing. As long as women remain a minority in the field of science, the men would overshadow their achievements even in the New Generation of women scientists causing a repeat of the Matilda Effect. None of these women have the name recognition and financial success as Albert Einstein, Nikola Tesla, and Thomas Edison, despite making comparably life-changing discoveries, with the repeated pattern of women under acknowledged for their work, it’s hard to be optimistic that much would change in the near future, however, we can do our part to continue to highlight women’s achievements and the definitely acknowledge the hardships that come with it.

\clearpage