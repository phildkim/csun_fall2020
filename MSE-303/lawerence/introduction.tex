\fancyhead[R]{\textsc{Lawrence Summers' Comments}}
\rfoot{\textsc{Page \thepage}}

\par
Women representation in the labor force has improved over the decades with now accounting nearly 
half the workforce. Such an improvement to the general workforce is disproportionately reflected 
in engineering and science fields. There are several hypotheses as to why there is a wide gap in 
these fields while more nurturing fields such as teaching, nursing, and accounting. The arguments 
given to the disparity range from sociological structures to innate intellectual capacity. We will 
examine the former Harvard University President, Lawrence Summers’ comments on women in the fields 
of science that have sparked controversy. The movement for 	equal representation for women across 
different career fields are worth striving for, yet the journey to get to the the end goal of equal 
representation of women in science will unfortunately take more time than we would like mainly due 
to systemic sexism prevalent from early child development. 

\par
Lawrence Summers was the former President of Harvard who was pressured to resign after making 
controversial remarks concerning the underrepresentation of women in the fields of science. 
Summers has quite an extensive educational background having attended Massachusetts Institute 
of Technology at an early age of 16, which he then proceeded to attend Harvard University as 
a graduate earning his Ph.D. in 1982, and became one of youngest professors at age of 28. 
As for his professional career, he worked closely with Clinton’s administration as Treasury 
Secretary~\cite{ustreasury} and Obama administration as director of the National Economic 
Council~\cite{economicconcil}. Despite his impressive educational and professional background, 
his comments regarding why women are underrepresented in sciences took his career in a downward turn.	

\par
An article written by PBS NewsHour, \say{Harvard President Summers’ Remarks About Women in Science, 
Engineering},\cite{summers} appears that during his time as President at Harvard, Summers found 
himself in a controversial predicament still discussed today even 15 years after. A problem that 
caught the interest of Summers, led to him to address why women are underrepresented particularly 
in science and engineering. In doing so, he wanted to, \say{\dots attempt to adopt an entirely 
positive, rather than normative approach, \dots}~\cite{summers} to provoke his audience. Summers 
attempted to answer the reasons why women are underrepresented in the fields of science with three 
hypotheses ranked according to importance: (1) high-powered job hypothesis, (2) availability of 
aptitude hypothesis, and (3) socialization and patterns of discrimination hypothesis.

\par
On top of Summers’ list, the high-powered job hypothesis is the main cause for the lack of females 
in science and engineering fields as he said, \say{\dots who wants to do high-powered intense work}\cite{summers}, 
implying that males tend to favor high-powered intense work. He questions why women in their mid 20s 
have less high-powered intense work requiring 80+ hours per week compared to their male counterparts. 
Someone who had worked with him closely during Treasury, reported that of all the 22 Harvard Business 
graduates who were females, only three had a full time position. The numbers show that women more often 
leave jobs for familial obligations, not due to lack of ability. Summers' hypothesis exposes at what 
stage women start to leave the job market, but not exactly explaining why that is.

\par
Summers’ second hypothesis: availability of aptitude hypothesis caused the most controversial debate. 
History Professor Susan Williams implies that Summers is saying that women are genetically different 
from men and lacking the natural ability of men in science and engineering fields~\cite{williams}. 
In her speech, Williams expressed discontent that someone in Summers’ position would say a ridiculous 
comment. She then talks about how Aristotle believed women are defective and Summers' way of thinking 
is similar to Aristotle. History has proven men like Summers have consistently thought of women as 
inferior to men and points to greater male variability hypothesis as evidence because men score higher 
than women especially in math and science.	

\par
The last hypothesis on the lack of women in science is due to socialization and patterns of discrimination. 
He describes that girls are socialized towards nursing while boys are towards building bridges. However, 
according to him, \say{\dots there are issues of intrinsic aptitude, and particularly of the variability 
of aptitude, and that those considerations are rein-forced by what are in fact lesser factors involving 
socialization and continuing discrimination.}\cite{summers} He believes socialization and patterns of 
discrimination hypothesis is not the biggest factor, rather it is more family desires and employers’ desire.

\par
As a computer science major, I can attest first hand that 90\% of my computer science classmates are males. 
Although I do believe Summers' first and third hypotheses have merit, I believe his most controversial 
hypothesis is still connected to the socialization of females during developmental years. There may be 
underlying psychological effects of being questioned more as a young girl than a young boy, which in turn 
promotes confidence in the case of males and discourages confidence in the case of females. Though I do 
not have supporting proof in this, however, the power of the mind of how an individual is socialized does 
make an effect at the margins and in turn can affect the objective results in aptitude tests.	

\par
After being pressured to resign his position as President at Harvard, Summers did apologize for his derogatory 
comments on the availability of aptitude hypothesis. Summers had an opinion that may or may not be valid and 
everyone is entitled to their own opinion. I don’t believe saying his opinion should lead to his resignation 
as long as it does not affect his day to day duties as President at Harvard. If we think anyone who thinks 
differently from us needs to resign a position, we are violating their freedom of speech and are upholding 
cancel culture. A person's opinion may differ from one stage of life to another and canceling everything 
they've done because of one opinion is not consistent with the ideals that build our country.	

\clearpage