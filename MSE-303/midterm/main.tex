\documentclass[11pt,a4paper]{article}
\usepackage[top=.9in,left=.5in,right=.5in,bottom=.9in]{geometry}
\usepackage{hyperref}
\begin{document}
  \begin{center}
    \textsc{\LARGE Midterm}
  \end{center}
  \begin{enumerate}
    \item Understand the Matilda Effect -\ what it is and who was affected by it
    \begin{itemize}
      \item 
    \end{itemize}
    \item The different inventions that women helped bring forth
    \begin{itemize}
      \item 
    \end{itemize}
    \item Madame Curie Complex -\ what it is and which prominent women are being studied
    \begin{itemize}
      \item 
    \end{itemize}
    \item Understand the various women studied in the video Agamede’s Legacy: History of Women of Women in Science and their place in history
    \begin{itemize}
      \item \url{https://www.youtube.com/watch?v=UzUkUFUVhTs}
    \end{itemize}
    \item Harvard University President Lawrence Summers- what he said that was controversial
    \begin{itemize}
      \item high-powered job hypothesis
      \item availability of aptitude hypothesis
      \item socialization and patterns of discrimination hypothesis
    \end{itemize}
    \item The Double Bind -\ what it is, who suffered from it and why
    \begin{itemize}
      \item 
    \end{itemize}
    \item Understand the different acts enacted through US history to help improve women’s place in society and whether it worked or not.
    \begin{itemize}
      \item 
    \end{itemize}
    \item The significance of the Harvard Observatory and all members involved
    \begin{itemize}
      \item 
    \end{itemize}
    \item Understand the evolution of women in science throughout history
    \begin{itemize}
      \item 
    \end{itemize}
  \end{enumerate}
  \clearpage
  \begin{itemize}
    \item \textbf{Madame Marie Curie}:\ was a Polish and naturalized-French physicist and chemist who conducted pioneering research on radioactivity. 
    She was the first woman to win a Nobel Prize, the first person and the only woman to win the Nobel Prize twice, and the only person to win the Nobel 
    Prize in two scientific fields. She shared the 1903 Nobel Prize in Physics with her husband Pierre Curie and physicist Henri Becquerel, for their 
    pioneering work developing the theory of ``radioactivity''. Using techniques she invented for isolating radioactive isotopes, she won the 1911 Nobel 
    Prize in Chemistry for the discovery of two elements, polonium and radium.
    \item \textbf{Maria Goeppart Mayer}:\ was a German-born American theoretical physicist, and Nobel laureate in Physics for proposing the nuclear shell 
    model of the atomic nucleus. She was the second woman to win a Nobel Prize in physics, the first being Marie Curie.
    \item \textbf{Agnes Pockles}:\ was a German pioneer in chemistry. Her work was fundamental in establishing the modern discipline known as surface science, 
    which describes the properties of liquid and solid surfaces.
    \item \textbf{C.S. Wu’s}:\ a Chinese-American experimental physicist who made significant contributions in the field of nuclear physics. Wu worked on the 
    Manhattan Project, where she helped develop the process for separating uranium into uranium-235 and uranium-238 isotopes by gaseous diffusion. She is best 
    known for conducting the Wu experiment, which proved that parity is not conserved. This discovery resulted in her colleagues Tsung-Dao Lee and Chen-Ning 
    Yang winning the 1957 Nobel Prize in Physics, while Wu herself was awarded the inaugural Wolf Prize in Physics in 1978.
    \item \textbf{Matilda Joslyn Gage}:\ was a women's suffragist, Native American rights activist, abolitionist, freethinker, and author. She is the eponym 
    for the Matilda Effect, which describes the tendency to deny women credit for scientific invention.
    \item \textbf{Si-ling-chi}:\
    \item \textbf{Constance Greene}:\
    \item \textbf{Lillian Gilbreth}:\ was an American psychologist, industrial engineer, consultant, and educator who was an early pioneer in applying psychology 
    to time-and-motion studies.
    \item \textbf{Florence Nightingale}:\ 
    \begin{itemize}
      \item[] \url{https://en.wikipedia.org/wiki/Florence_Nightingale} 
    \end{itemize}
    \item \textbf{Hildegard of Bingen}:\
    \begin{itemize}
      \item[] \url{https://en.wikipedia.org/wiki/Hildegard_of_Bingen}
    \end{itemize}
    \item \textbf{Sister Maria Celeste}:\
    \begin{itemize}
      \item[] \url{https://en.wikipedia.org/wiki/Maria_Celeste} 
    \end{itemize}
    \item \textbf{Trotula di Ruggiero}:\ 
    \begin{itemize}
      \item[] \url{https://en.wikipedia.org/wiki/Trotula} 
    \end{itemize}
    \item \textbf{Hypatia of Alexandria}:\ 
    \begin{itemize}
      \item[] \url{https://www.youtube.com/watch?v=n1mwZrVJ-TI}
      \item[] \url{https://www.youtube.com/watch?v=0yIEhUSrGCM&feature=youtu.be}
    \end{itemize}
    \item \textbf{Women of the Harvard Observatory}:\
    \item \textbf{Maria Mitchell}:\ as an American astronomer, librarian, naturalist, and educator. In 1847, she discovered a comet named 1847 VI (modern designation C/1847 T1) 
    that was later known as ``Miss Mitchell’s Comet'' in her honor.
    \item \textbf{Lise Meitner}:\
    \begin{itemize}
      \item [] \url{https://www.youtube.com/watch?v=OVT0suIlzXQ&feature=youtu.be}
    \end{itemize}
    \item \textbf{Margaret Cavendish}:\ 
    \begin{itemize}
      \item[] \url{https://en.wikipedia.org/wiki/Margaret_Cavendish,_Duchess_of_Newcastle-upon-Tyne}
    \end{itemize}
    \item \textbf{Maria Gaetana Agnesi}:\ She was the first woman to write a mathematics handbook and the first woman appointed as a mathematics professor at a university.
    \item \textbf{Caroline Herschel}:\ 
    \begin{itemize}
      \item[] \url{https://www.youtube.com/watch?v=ocGHWf1sX_Q&feature=youtu.be}
      \item[] \url{https://www.youtube.com/watch?v=0yIEhUSrGCM&feature=youtu.be}  
    \end{itemize} 
    \item \textbf{Laura Bassi}:\ she was the second woman in the world to earn the degree of Doctor of Philosophy (after the philosopher Elena Cornaro Piscopia, who had received 
    doctorate in 1678) and the first woman to have doctorate in science.
    \item \textbf{Emily Noether}:\ was a German mathematician who made many important contributions to abstract algebra. She has also a famous theorem in mathematical physics known 
    as Noether's theorem.
    \item \textbf{James Barry}:\ 
    \begin{itemize}
      \item[] \url{https://en.wikipedia.org/wiki/James_Barry_(surgeon)} 
    \end{itemize}
    \item \textbf{Elizabeth Blackwell}:\ was a British physician, notable as the first woman to receive a medical degree in the United States, and the first woman on the Medical Register 
    of the General Medical Council.
    \item \textbf{Dorothy Hodgkin}:\
    \begin{itemize}
      \item[] \url{https://www.youtube.com/watch?v=JoXdBwO-pMc&feature=youtu.be}
      \item[] \url{https://www.youtube.com/watch?v=zrryPaMnGVs&feature=youtu.be} 
    \end{itemize}
  \end{itemize}
\end{document}