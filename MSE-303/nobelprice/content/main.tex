
\par
The Nobel Prize started in the early 19th century because of a gentleman by the name of Alfred Nobel who had 
substantial worth due to his many inventions. His interests in physics, chemistry, physiology, literature, 
and peace work are the reasons why the Nobel Prize is divided into five different categories. As one of his 
wills before he passed away, his wish was to give out his fortune as prizes to those who can bestow the 
greatest among humankind in the five categories he was most interested in. This prestigious ceremony overtime 
has been awarded to 950 people and organizations, however, there is a significant difference in the number of 
women laureates compared to men laureates.

\par
My initial response to the prompt of whether or not the Nobel Prize should implement quotas to improve the 
gender balance was a hard no. This reasoning was due to the fact that I believed science should not be 
diluted for the sake of gender equality, but rather for the sake of the quality of science itself. After 
watching the videos and reading the articles about the struggles women face in pursuing their scientific 
endeavour, it seems illogical not to implement quotas. It’s disheartening to read and hear about the women 
who were mistreated and undervalued as scientists throughout history, such as Marie Curie and Lise Meitner.\cite{kopievanpbs}~\cite{jardins}
These women were very much in their own right the top scientists in their field, but were recognized mainly for 
their connection to men in their lives-in the case of Marie Curie, her husband Pierre Curie, and Lisa Meitner 
her colleague Otto Hahn.

\par
I had the impression that without a doubt the Nobel Prize always goes to the scientist with the most groundbreaking 
discovery for the advancement of humanity. It turns out there’s more than just pure science, politics and fame has 
fallen into the mix of this prestigious prize, which seems to be the main reasoning why women are underrepresented 
in the Nobel Prize awards. In the case of both Marie Curie and Lisa Meitner, they are both known to be reserved in 
their temperament. Even with fame, Marie Curie did not view her discoveries as a means to make money, but rather 
they \say{must belong to science.}\cite{jardins}

\par
The women scientists in history, including Dorothy Hodgkin in addition to Curie and Meitner demonstrate that Larry 
Summers’ hypothesis concerning women being biologically inferior in science than men is construed.~\cite{dorothy1}~\cite{dorothy2}~\cite{jardins}
It doesn’t tell the whole story about the struggle women have had in the political realm and in their pursuit to obtain 
their education. When they do enter the science field, according to Julie Des Jardins in her book, \say{The Madame Curie 
Complex: The Hidden History of Women in Science}, women in the science fields are professionally viewed as \say{assistants, 
housekeepers, and interchangeable parts},\cite{jardins} capable of merely cleaning equipment in the labs and collecting 
plants for the real botanists who were men. 

\par
The number of female science students during their undergraduate years make up 60\% of all students, and half of them drop 
out of the field during their professional lives.\cite{drjemison} Based on the adversity Meitner and Curie faced in getting 
their education during the late 19th century, there’s no wonder women were severely underrepresented in the STEM field, however, 
there still seems to be an underlying struggle even in today’s age for women scientists, especially those of color with the lack of role 
models. It seems that exposure to the sciences, work atmosphere, and work-family life balance is the culprit of the underrepresentation 
of women in STEM professions. First, although more young girls are exposed to sciences in this day in age, it’s still not widely 
encouraged in social standards reflected in the toys made for girls. Second, the work atmosphere in a male-dominated industry with 
its race to political gain and fame may not suit a reserved female scientist’s personality. Finally, the work-family life balance 
is a struggle that professional females in any industry faces. For all these reasons working against women in the STEM fields, the 
Nobel Prize and scientific organizations should support and encourage women in their research by implementing a quota dedicated to 
achievements made by female scientists.