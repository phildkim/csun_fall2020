\noindent
\say{The percentage of Nobel Prizes winners in science have historically been geared more towards men. As such, some have called for 
the various Nobel committees to instill quotas ensuring a certain percentage of women to receive the prize. Take a position on this 
proposition and explain in 2–3 pages (double spaced) \textbf{why you do or do not think the Nobel Prize should implement quotas to 
improve the gender balance.}}
\clearpage

\par
Alfred Nobel was an inventor, entrepreneur, scientist and businessman who also wrote poetry and drama. His varied interests are reflected 
in the prize he established and which he lay the foundation for in 1895 when he wrote his last will, leaving much of his wealth to the 
establishment of the prize. Since 1901, the Nobel Prize has been honoring men and women from around the world for outstanding achievements 
in physics, chemistry, physiology or medicine, literature and for work in peace. Alfred Nobel signed his last will in Paris on November 27, 1895. 
He specified that the bulk of his fortune should be divided into five parts and to be used for prizes in physics, chemistry, physiology or medicine, 
literature and peace to \say{those who, during the preceding year, shall have conferred the greatest benefit to humankind.}\cite{alfrednoble}

\par
Total of 923 Laureates and 27 organizations has been awarded the Nobel Price since 1901 and 2019. Only 54 out of 923 Laureates, 5\% of whom
are women.\cite{alfrednoble}

\par
In terms of the most prestigious awards in STEM fields, only a small proportion have been awarded to women. Out of 210 laureates in 
Physics, 181 in Chemistry and 216 in Medicine between 1901 and 2018, there were only three female laureates in physics, five in 
chemistry and 12 in medicine. Factors proposed to contribute to the discrepancy between this and the roughly equal human sex ratio 
include biased nominations, fewer women than men being active in the relevant fields, Nobel Prizes typically being awarded decades 
after the research was done (reflecting a time when gender bias in the relevant fields was greater), a greater delay in awarding 
Nobel Prizes for women's achievements making longevity a more important factor for women (Nobel Prizes are not awarded posthumously), 
and a tendency to omit women from jointly awarded Nobel Prizes. Despite these factors, Marie Curie is to date the only person awarded 
Nobel Prizes in two different sciences (Physics in 1903, Chemistry in 1911); she is one of only three people who have received two 
Nobel Prizes in sciences. 

\par
\say{The Path to Nuclear Fission},\cite{kopievan}
Lise Meitner faces gender prejudice while Otto Hahn career blossomed. Even after she finally reached the same title as Hahn, she was
well under paid compared to Hahn. Despite all the hardship she faced, the journey of discovering new elements with Hahn began. She dealt
with gender prejudice, ethnicity prejudice, World Wars, Hitler, depression,


\par
Programs for minorities and women have generally been assumed to include minority women, but in fact minority women fall in the cracks 
between the two, also known as \say{double bind}.\cite{doublebind}
Minority males and majority females must come to realize that a demand for the minority woman to make a choice places her in an untenable
position since she can deny neither the fact of her race or ethnic identity nor her gender. Nor can she avoid the problems associated with 
both. 
