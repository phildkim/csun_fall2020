\begin{enumerate}
  \item \textsc{Introduction}
  \begin{itemize}
    \item What is Nobel Prize? Who? Why?
    \item Since 1901, Nobel Prize has been awarded 597 times to 950 people and organizations. 
    \item 53 women have been awarded the Nobel Price. 1901(4) -\ 2019(24)
  \end{itemize}
  \item \textsc{Body Paragraph}
  \begin{itemize}
    \item What is the problem? 
    \item Reference \say{Double Bind: The Price of Being a Minority Woman in Science}\cite{doublebind} 
    \item Reference \say{The Path to Nuclear Fission, The Story of Lise Meitner and Otto Hahn}\cite{kopievanpbs}
    \item Problem in Nobel Price: Laureates -\ 923, organizations -\ 27, only 54 Laureates are women (5\%)\cite{alfrednoble}
  \end{itemize}
  \item \textsc{Conclusion}
  \begin{itemize}
    \item Nobel Prize should not implement quotas to improve the gender balance.
    \item Reason being, quotas can potentially overlook better qualified candidates.
    \item Another solution is eliminate biased selection by removing names of nominees.
    \item Also, ensure that Nobel Committees has an equal gender ratio.
  \end{itemize}
\end{enumerate}
\clearpage
\fancyhead[R]{\textsc{Introduction}}
\par
Nobel Prize is known to most of us as a prestigious award for contributions in these five fields: 
chemistry, literature, peace, physics, and physiology.

 

% Alfred Nobel was an inventor, entrepreneur, scientist and businessman who also wrote poetry and drama. 
% His varied interests are reflected in the prize he established and which he lay the foundation for in 
% 1895 when he wrote his last will, leaving much of his wealth to the establishment of the prize. Since 
% 1901, the Nobel Prize has been honoring men and women from around the world for outstanding achievements 
% in physics, chemistry, physiology or medicine, literature and for work in peace. Alfred Nobel signed his 
% last will in Paris on November 27, 1895. He specified that the bulk of his fortune should be divided into 
% five parts and to be used for prizes in physics, chemistry, physiology or medicine, literature and peace 
% to \say{those who, during the preceding year, shall have conferred the greatest benefit to humankind.
% }\cite{alfrednoble}

% \clearpage
% \fancyhead[R]{\textsc{Problems with Nobel Prize}}
% \par
% Laureates -\ 923, organizations -\ 27, only 54 Laureates are women (5\%).\cite{alfrednoble}
% \say{The traditional male domination of the science fields has made the attainment of and participation 
% in science careers for women difficult at best. The price of a professional science career is therefore 
% significantly higher for a woman. The demands on women to assume family-related responsibilities are not 
% thought to be compatible with study for or work in these traditionally male professions. The mode of academic 
% preparation and work-style have been developed around traditional majority male lifestyles which differ 
% substantially from the varied life patterns of women. Role stereotyping and sex discrimination add to 
% the personal costs of women who seek to fulfill career goals as scientists, engineers, or biomedical 
% professionals.}\cite{doublebind}

% \clearpage
% \fancyhead[R]{\textsc{Conclusion}}
% \par
% Don't implement quotas to improve gender balance because doing so could possibly override potentially
% better ideas by forcefully selecting lesser nominee due to quotas. I believe another solution may benefit both
% the purpose of what Alfred Nobel wanted as his last will to honor men and women for their outstanding achievements
% and also to increase the percentage for women as winners. To increase the numbers of women as nominees, biased
% nominations need to be eliminated by simply having equal gender ratio as Nobel Committees. To further eliminate
% any bias selection, keep all the nominees names anonymous for the obvious reasons of pre-judgement by names.
