\begin{enumerate}
  \item \textsc{Introduction}
  \begin{itemize}
    \item What is Nobel Prize? Who? Why?
    \item Since 1901, Nobel Prize has been awarded 597 times to 950 people and organizations. 
    \item 53 women have been awarded the Nobel Price. 1901(4) -\ 2019(24)
  \end{itemize}
  \item \textsc{Body Paragraph}
  \begin{itemize}
    \item What is the problem? 
    \item Reference \say{Double Bind: The Price of Being a Minority Woman in Science}\cite{doublebind} 
    \item Reference \say{The Path to Nuclear Fission, The Story of Lise Meitner and Otto Hahn}\cite{kopievanpbs}
    \item Problem in Nobel Price: Laureates -\ 923, organizations -\ 27, only 54 Laureates are women (5\%)\cite{alfrednoble}
  \end{itemize}
  \item \textsc{Conclusion}
  \begin{itemize}
    \item Nobel Prize should not implement quotas to improve the gender balance.
    \item Reason being, quotas can potentially overlook better qualified candidates.
    \item Another solution is eliminate biased selection by removing names of nominees.
    \item Also, ensure that Nobel Committees has an equal gender ratio.
  \end{itemize}
\end{enumerate}
% The percentage of Nobel Prizes winners in science have historically been geared more towards men. 
% As such, some have called for the various Nobel committees to instill quotas ensuring a certain 
% percentage of women to receive the prize. Take a position on this proposition and explain in 2–3 
% pages (double spaced) why you do or do not think the Nobel Prize should implement quotas to improve 
% the gender balance.
\clearpage
\fancyhead[R]{\textsc{Introduction}}
\par
The Nobel Prize originated start of 19th century because of a gentlemen, by the name of Alfred Nobel
who had substantial worth due to his many inventions. His interests in physics, chemistry, physiology,
literature, and peace work are the reasons why the Nobel Prize is divided into five different categories.
As one of his will before he passed away, his wish was to give out his fortune as prizes to those who can 
bestow the greatest among humankind in the five categories he was most interested. The process of selecting
a winner can seem quite confessing, as it begins with about 300 nominees (aka laureates) whom are selected
by Nobel Committee and other individuals working in the relevant field. Then the nominated laureates will
be awarded and revealed 50 years later. This prestigious ceremony overtime has been rewarded to 950 people
and organizations, however, there is a significant difference in women laureates compared to men laureates.
~\cite{alfrednoble}

\par
Although the number of women laureates has increased since 1901, the numbers are increasing rather too slowly.
An explanation in early 19th century has to do with gender bias, but explaining why it's still low numbers today
has more explanation. 

\clearpage
\fancyhead[R]{\textsc{Double Bind, Double Standard}}
\par
\say{The traditional male domination of the science fields has made the attainment of and participation 
in science careers for women difficult at best. The price of a professional science career is therefore 
significantly higher for a woman. The demands on women to assume family-related responsibilities are not 
thought to be compatible with study for or work in these traditionally male professions. The mode of academic 
preparation and work-style have been developed around traditional majority male lifestyles which differ 
substantially from the varied life patterns of women. Role stereotyping and sex discrimination add to 
the personal costs of women who seek to fulfill career goals as scientists, engineers, or biomedical 
professionals.}\cite{doublebind}
\par
Double bind is similar to what psychologist might call double standard and unfortunately our society still
has a lot of double bind and double standard. Women are often portrayed 

\clearpage
\fancyhead[R]{\textsc{Conclusion}}
\par
Don't implement quotas to improve gender balance because doing so could possibly override potentially
better ideas by forcefully selecting lesser nominee due to quotas. I believe another solution may benefit both
the purpose of what Alfred Nobel wanted as his last will to honor men and women for their outstanding achievements
and also to increase the percentage for women as winners. To increase the numbers of women as nominees, biased
nominations need to be eliminated by simply having equal gender ratio as Nobel Committees. To further eliminate
any bias selection, keep all the nominees names anonymous for the obvious reasons of pre-judgement by names.
